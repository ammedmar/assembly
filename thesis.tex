\documentclass[12pt,oneside]{book}
\usepackage{mystyle}
\usepackage{subfiles}

\begin{document}

	\chapter*{Introduction}
	\subfile{sec/abstract}
	\chapter{Simplicial sets and $\S$-coalgebras}

	\paragraph{Convention.}\hspace*{-8pt}The term chain complex will be reserved for a homologically graded differential graded abelian group. The category of chain complexes, denoted by $\Ab_\bullet$, is enriched over itself, i.e. $\Hom_{\Ab_\bullet}(C,C')\in\Ab_\bullet$ for every pair $C,\,C'\!\in\Ab_\bullet$. In terms of this enrichment, chain maps correspond to $0$-degree cycles, while chain homotopy equivalent morphisms correspond to homologous chains.

	\section{Operads, coalgebras and comodules}
	\subfile{sec/Operads-coalgebras-and-comodules}

	\section{The operad $\S$}
	\subfile{sec/The-operad-S}

	\section{Simplicial sets and $\S$-coalgebras}
	\subfile{sec/Simplicial-sets-and-S-coalgebras}

	\chapter{Abelian sheaves and $\S$-comodules}

	\section{Sheaf theory of posets}
	\subfile{sec/Sheaf-theory-of-posets}

	\section{Simplicial complexes and Ranicki duality}
	\subfile{sec/Duality}

	\section{Topological manifolds and $\S$-comodules}
	\subfile{sec/Topological-manifolds-and-S-comodules}

	\appendix
	\chapter{Categorical Background}
	\subfile{sec/Appendix}

	\bibliographystyle{plain}
	\bibliography{mybib}
	\nocite{*}
\end{document}

