%\documentclass[thesis.tex]{subfiles}

%\begin{document}

\subsection{Topological manifolds and $\S$-comodules}

In this section, as Theorem \ref{sheaves fully-faithful to comodules}, the second of the two main technical results of this work is presented. It states that the category of complexes of sheaves over an ordered simplicial complex $X$ with values in $\Ab$ embeds as a differential graded full subcategory of the category of comodules over $\chains(X)$ as an $\S$-coalgebra.

This theorem is used to relate the algebraic surgery theory of Ranicki with comodules over $E_\infty$-coalgebras. In particular, Theorem \ref{existence and uniqueness for hom mnfds} and Theorem \ref{existence and uniqueness for top mnfds} provide existence and uniqueness statements for homology manifold structures and topological manifold structures on the homotopy type of a Poincar\'e duality regular pseudomanifold, using comodules on its $\S$-coalgebra of chains.

\begin{notation}
The tensor product over $X$ with the complex of cosheaf $\chains$ defines a functor from $\Sh(X,\Ab)_\bullet$ to $\Ab_\bullet$, see Definition \ref{tensor product of functors} and Definition~\ref{sheaf of cochains and cosheaf of chains} for unfamiliar terminology. For $\D_\bullet$ a complex of sheaves, $\D_\bullet\tensor_X\chains$ is given by
$$\bigoplus_{\sigma\in X}\D_\bullet[\sigma]\tensor\chains[\sigma]/\sim$$
with $d_\tau\,\tensor\chains[\sigma\leq\tau](c_\sigma)\sim \D_\bullet[\sigma\leq\tau](d_\tau)\tensor\,c_\sigma$ and differential graded structure induce from the tensor product of chain complexes. It is isomorphic as abelian group to
$$\bigoplus_{\sigma\in X}\D_\bullet[\sigma]\tensor\sigma,$$ and elements of the form $d\,\tensor\sigma$ for some $\sigma\in X$ will be referred to as \textbf{canonical representatives} of elements in $\D_\bullet\tensor_X\chains$. Compare with the proof of Lemma \ref{T^2 = id}.

Notice that given $F:\D_\bullet\to\D'_\bullet$ a morphism of complexes of sheaves, the induced morphism is given in terms of canonical representatives by $$f(d\tensor\sigma)=F[\sigma](d)\tensor\sigma.$$
\end{notation}

\begin{lemma}\label{functor tensor product lift to comodules}
The functor $-\tensor_X\chains$ lifts along the forgetful functor to the category of comodules over the $\S$-coalgebra $\chains(X)$. Diagrammatically,
$$\xymatrix @C=2pc @R=2pc{\ & \coMod^{\S}_{\chains(X)} \ar^{\text{forget}}[d]\\
\Sh(X,\Ab)_\bullet \ar[r]_-{-\stackrel[X]{}{\tensor}\chains}\ar@{-->}[ur] & \Ab_\bullet.}$$
\begin{proof}
For every $\sigma\in X$ the complex $\chains[\sigma]=\chains(\cl\sigma,\partial)$ is an $\S$-coalgebra naturally, so the functor $\chains:X\to\Ab_\bullet$ can be lifted along the forgetful functor to $\chains:X\to\coMod_{\chains(X)}^{\S}$ with structure maps $$\S(k)\tensor\chains[\sigma]\to\chains[\sigma]^{\tensor k}\to\chains[\sigma]\tensor\chains(X)^{\tensor(k-1)}.$$
The functor tensor product $\D_\bullet\tensor_X\chains$ inherits a $\S$-comodule structure over $\chains(X)$ from its second factor.
\end{proof}
\end{lemma}

\begin{remark}
By forgetting structure, $\D_\bullet\tensor_X\chains$ is also a comodule over $\chains(X)$ thought of as an $\S(2)$ coalgebra, i.e. a coalgebra over the operad generated by the arity $2$ part of the operad $\S$. The coaction of the generator $(\dotsc,1,2,1)$ of $\S(2)$ of degree $k$ will be denoted by $\nabla_k$ and, according to Lemma \ref{functor tensor product lift to comodules}, it is defined for any class $d\tensor c\in\D_\bullet\tensor_X\chains$ by $$\nabla_n(d\tensor c)=d\tensor\Delta_n(c),$$
with the notation $\Delta_n$ introduced in \ref{notation for Deltas}.
\end{remark}

\begin{theorem}\label{sheaves fully-faithful to comodules}
The differential graded functor
$$-\stackrel[X]{}{\tensor}\chains:\Sh(X,\Ab)_\bullet\longrightarrow\coMod^{\S(2)}_{\chains(X)}$$
is full and faithful.
\begin{proof}
Let $F:\D_\bullet\to\D'_\bullet$ be a morphisms of complexes of sheaves and assume the morphism induced by the functor $-\tensor_X\chains$ is $0$. Using canonical representatives, this implies that the abelian group homomorphism
$$\bigoplus_{\sigma\in X}(F[\sigma]\tensor\id_{\sigma}):\bigoplus_{\sigma\in X}\D_\bullet[\sigma]\tensor\sigma\longrightarrow\bigoplus_{\sigma\in X}\D'_\bullet[\sigma]\tensor\sigma$$ is $0$, hence $F[\sigma]=0$ for each $\sigma\in X$, i.e. $F=0$.

Given an $\S$-$\chains(X)$-comodule map $f:\D_\bullet\tensor_X\chains\to\D_\bullet'\tensor_X\chains$ one needs to construct a sheaf map inducing it. Let $d_{\sigma}\tensor\sigma$ be a canonical representative and write its image in terms of canonical representatives $f(d_{\sigma}\tensor\sigma)=\sum_{\tau\in X}d'_{\tau}\tensor \tau$. Since $\nabla_n\circ f=(f\tensor \id)\circ\nabla_n$ one has for all $n\geq0$ that
$$\xymatrix{d_\sigma\tensor\sigma \ar@{|->}[r]^-f \ar@{|->}[d]_{\nabla_n} & \sum_{\tau\in X}d'_\tau\tensor\tau \ar@{|->}[d]^{\nabla_n}\\
d_\sigma\tensor\Delta_n\sigma \ar@{|->}[r]_-{f\tensor\,\id} & (?)_n\sim \sum_{\tau\in X}d'_\tau\tensor\Delta_n\tau.}$$
The above equation will be used to show that $d'_\tau=0$ for all $\tau\neq\sigma$. Let $n$ be the largest $|\tau|$ so $d'_\tau\neq0$ and assume $n>|\sigma|$, then
$$(?)_n=0\sim\sum_{|\tau|=n}d'_\tau\tensor\tau\tensor\tau$$
so $d'_{\tau}=0$ for all $\tau$ of dimension $n$. Iterating this argument one has that $d'_\tau=0$ for all $\tau$ of dimension greater than $|\sigma|$. For $n=|\sigma|$ one has $$(?)_n=\sum_{\tau\in X}d'_\tau\tensor\tau\tensor\sigma\sim\sum_{|\tau|=|\sigma|}d'_\tau\tensor\tau\tensor\tau$$
so $d'_\tau=0$ for all $\tau\neq 0$. For each $\sigma\in X$ define the chain map
$$\begin{array}{cc}
f_\sigma: & \D_\bullet[\sigma]\to\D_\bullet[\sigma] \\
 & \,e_\sigma\mapsto e'_\sigma.
\end{array}$$
If the above collection of chain maps defines a morphism from $\D_\bullet$ to $\D_\bullet'$ then it induces $f$, so it needs to be shown that for every $\iota:\rho\to\sigma$ one has $\D_\bullet'[\iota]\circ f_{\sigma}=f_{\rho}\circ \D_\bullet[\iota]$. Assume for an induction argument that this holds for all morphisms $\iota:\rho\to\sigma$ with $|\sigma|<n$. For any simplex $\sigma=[v_0,\dotsc,v_n]$ denote its $i$-th face by $\sigma_i=[v_0,\dotsc,\hat{v_i},\dotsc,v_n]$ and $\iota^i:\sigma_i\to\sigma$. The induction assumption and the functoriality of sheaves imply that it suffices to show that
\begin{equation}\label{to be verified}
\D'_\bullet[\iota^i]\circ f_{\sigma}=f_{\sigma_i}\circ \D_\bullet[\iota^i]
\end{equation}
for all simplices $\sigma$ of dimension $n$ and $i=0,\dotsc,n$.

For any such $\sigma\in X$ and $d\in\D_\bullet[\sigma]$ consider the following diagram
\begin{equation}\label{eq [1,...,n]}
\xymatrix{d\tensor\sigma \ar@{|->}[r]^-f \ar@{|->}[d]_{\nabla_0} & f_\sigma(d)\tensor\sigma \ar@{|->}[d]^{\nabla_0}\\
d\tensor\Delta_0\sigma \ar@{|->}[r]_-{f\tensor\,\id} & (?)\sim f_\sigma(d)\tensor\Delta_0\tau.}
\end{equation}
Recall from Example \ref{Delta_0 (no signs)} that $\Delta_0[0,\dotsc,n]=\sum_{i}[0,\dotsc,i]\tensor[i,\dotsc,n]$ so projecting $\D'_\bullet\tensor_X\chains\tensor\chains(X)$ onto $\D'_\bullet\tensor_X\chains\tensor\,[v_{n-1},v_n]$ makes the equation in (\ref{eq [1,...,n]}) be
$$f\big(d\tensor\iota^n_\bullet\sigma_n\big)\sim f_\sigma(d)\tensor\iota^n_\bullet\sigma_n$$
which implies
$$\big(f_{\sigma_n}\circ \D_\bullet[\iota^n]\big)(d)=\big(\D'_\bullet[\iota^n] \circ f_{\sigma}\big)(d).$$
A completely analogous argument using $T\nabla_0$ verifies equation (\ref{to be verified}) for $i=0$. For all $0<i<n$ one consider the following diagram associated to $\nabla_1$
\begin{equation}\label{eq [0,...,hat i,...,n]}
\xymatrix{d\tensor\sigma \ar@{|->}[r]^-f \ar@{|->}[d]_{\nabla_1} & f_\sigma(d)\tensor\sigma \ar@{|->}[d]^{\nabla_1}\\
d\tensor\Delta_1\sigma \ar@{|->}[r]_-{f\tensor\,\id} & (?)\sim f_\sigma(d)\tensor\Delta_1\tau.}
\end{equation}
Recall that $\Delta_1[0,\dotsc,n]=\sum_{i<j}\pm[0,\dotsc,i,j,\dotsc,n]\tensor[i,\dotsc,j]$ so projecting $\D'_\bullet\stackrel[X]{}{\tensor}\chains\tensor\chains(X)$ onto $\D'_\bullet\stackrel[X]{}{\tensor}\chains\tensor\,[v_{i-1},v_{i+1}]$ makes the equation in (\ref{eq [0,...,hat i,...,n]}) be
$$f\big(d\tensor\iota^i_\bullet\sigma_i\big)\sim f_\sigma(d)\tensor\iota^i_\bullet\sigma_i$$
which implies
$$\big(f_{\sigma_i}\circ \D_\bullet[\iota^i]\big)(d)=\big(\D'_\bullet[\iota^i] \circ f_{\sigma}\big)(d)$$ and completes the verification of equation $(\ref{to be verified})$.
\end{proof}
\end{theorem}

\begin{remark} One can consider the functor tensor product over $X$ of the cochain functor $\cochains$ and any $\E_\bullet\in\coSh(X,\A)_\bullet$. The analogue of the results above can be proven by similar arguments, but they will not be used in this work. In particular, the category of complexes of cosheaves over $X$ can be thought of as a full subcategory of the category of modules on the $\S$-algebra of cochains on $X$.
\end{remark}

\subsubsection{$L$-theory for sheaf-like $\S$-comodules}

\begin{definition}(Sheaf-like comodules and duality) An $\S$-comodules $D$ over $\chains(X)$ is said to be \textbf{sheaf-like} if it is isomorphic to one of the form $\D_\bullet\tc$ with $\D_\bullet$ a projective complex of sheaves with values on the category $\Ab_{\f}$ of finitely generated abelian groups. The Ranicki duality functor, Definition \ref{Ranicki duality functors}, induces by Lemma \ref{sheaves fully-faithful to comodules} a contravariant functor on the subcategory of sheaf-like comodules. Explicitly, let $D\cong\D_\bullet\tc$ and $D'\cong\D'_\bullet\tc$ be sheaf-like and $f\in\Hom_{\coMod}(D,D')$. By Lemma \ref{sheaves fully-faithful to comodules} there exists $F\in\Hom_{\Sh}(\D,\D')$ so that $F\tc=f$. Define, abusing notation, $\T f:\T D'\to\T D$ to be $$(\T F)\tc:(\T\D')\tc\to(\T\D)\tc.$$
\end{definition}

\begin{remark} \label{duality for comodules}
The natural transformation $\varepsilon:T^2\to\id_{\Sh}$ of Lemma \ref{T^2 = id} induces an analogous natural transformation for the duality of sheaf-like comodules. In particular, the chain complex $\Hom_{\coMod}(\T D,D)$ has an action of $\Sigma_2$ given by $f\mapsto \varepsilon_{D}\circ Tf$. For any sheaf-like comodule $D=\D\tc$, Theorem \ref{sheaves fully-faithful to comodules} gives a chain isomorphism $$\Hom_{\Sh}(\T\D,\D)^{\Sigma_2}\cong\Hom_{\coMod}(\T D,D)^{\Sigma_2}.$$
See \ref{homotopy fix points} for the definition of these complexes.
\end{remark}

\begin{definition}(Connective sheaf-like and Poincar\'e comodules)
A sheaf-like comodule is said to be \textbf{connective} if it is chain homotopy equivalent, as $\S$-comodule, to one which equals $0$ in negative degrees.

An $n$-\textbf{dimensional weak Poincar\'e comodule} is a pair $(D,\varphi)$ with $D$ a connective sheaf-like comodule and $\varphi$ a cycle of degree $n$ in $\Hom_{\coMod}(\T D,D)^{\Sigma_2}$ such that $\varphi_0$ is a homology isomorphism. (See Remark \ref{homotopy fix points} for unfamiliar notation.)

An $n$-\textbf{dimensional strong Poincar\'e comodule} is an $n$-dimensional weak Poincar\'e comodule $(D,\varphi)$ so that $\varphi_0$ is a chain homotopy equivalence of $\S$-comodules.
\end{definition}

\begin{notation}
Let $f:C\to C'$ be a chain map between chain complexes. Denote the \textbf{mapping cone} of $f$ by $\Cone(f)$, i.e. the chain complex $\Sigma C\oplus C'$ with boundary defined by $(\Sigma(c)+c')\mapsto \Sigma(\partial(c))+f(c)+\partial(c')$, where $\Sigma(-)$ stands for suspension.
\end{notation}

\begin{definition}(Cobordism)\label{cobordism definition}
A \textbf{weak cobordism} between $n$-dimensional weak Poincar\'e comodules $(D,\varphi)$ and $(D',\varphi')$ consists of a connective sheaf-like comodule $E$ with a degree $(n+1)$-chain $\phi\in\Hom_{\coMod}(\T E,E)^{\Sigma_2}$ and a couple of maps $f:D\to E$ and $f:D'\to E$ satisfying:
\begin{enumerate}
\item $\partial\phi=(f\circ\varphi\circ\T f)-(f'\circ\varphi'\circ\T f')$.
\item $(\phi/\!\varphi)_0\stackrel{\mathrm{def}}{=}\big(\phi_0+(\varphi_0\circ\T f)+(\varphi'_0\circ\T f')\big):\T E\to\Cone(f\oplus(-f'))$ is a homology isomorphism.
\end{enumerate}
\end{definition}
A \textbf{strong cobordism} between $n$-dimensional strong Poincar\'e comodules $(D,\varphi)$ and $(D',\varphi')$ is a weak cobordism between them such that $(\phi/\!\varphi)_0$ is a chain homotopy equivalence of $\S$-comodules.

\begin{example}
The central example of a weak Poincar\'e comodule for the applications of this work comes from Construction \ref{visible symmetric complex}. Let $X$ be a simply-connected regular pseudomanifold which is an $n$-dimensional Poincar\'e duality space. By Theorem \ref{sheaves fully-faithful to comodules} and Remark \ref{duality for comodules} the image by $-\tc$ of the visible symmetric complex, denoted $(\mathrm P,\varphi_{\mathrm{P}})$, would be an $n$-dimensional weak Poincar\'e comodule if $(\varphi_{\mathrm{P}})_0$ induces a homology isomorphism.

To see this is the case, recall that by definition one has for any complex of sheaves $\D_\bullet$ that $\D_\bullet\tc=\bigoplus_{\sigma\in X}\D_\bullet[\sigma]\tensor\chains[\sigma]/\!\!\sim$ and $\chains[\sigma]=\chains(\cl\sigma)$. Contracting each second factor one gets a chain map inducing an isomorphism in homology
\begin{equation}\label{contracting DxC to D}
\D_\bullet\tc\to\bigoplus_{v\in X^{(0)}}\D_\bullet[v]/\hat{\sim}
\end{equation}
with $\D_\bullet[v\leq\sigma](d_v)\,\hat{\sim}\,\D_\bullet[v'\leq\sigma](d_{v'})$ for every $\sigma\in X$. This assignment is functorial with a morphism of complexes of sheaf $F:\D\to\D'$ inducing the chain map $\bigoplus_{v\in X^{(0)}}F[v]$.

Recall from Remark \ref{pair subdivision related to dual cones} that $\Ps[\sigma]$ is chain homotopy equivalent to the simplicial chains on the closed dual cone of $\sigma$, in symbols $$\Ps[\sigma]\stackrel{che}{\to}\DC[\sigma]=\chains(\dc(\sigma));$$
and from Remark \ref{dual of pair subdivision related to relative dual cones} that $\T\Ps[\sigma]$ is chain homotopy equivalent to the relative simplicial cochains suspended by $|\sigma|$ of the closed dual cone of $\sigma$ modulo its boundary, in symbols
$$\T\Ps[\sigma]\stackrel{che}{\to}\Sigma^{|\sigma|}\!\cochains(\dc(\sigma),\partial\dc(\sigma)).$$
By the previous two observations, the morphism (\ref{contracting DxC to D}) and the definition of $\varphi_{\mathrm P}$ one has a commutative diagram
$$\xymatrix{\!\!\!\!\T\Ps\stackrel[X]{}{\tensor}\chains \ar[r]^-{(\varphi_{\mathrm{P}})_0} \ar[d] & \Ps\stackrel[X]{}{\tensor}\chains \ar[d]\\
 \cochains(X') \ar[r]_{-\cap[X']} & \chains(X')}$$
with vertical arrows representing homology isomorphisms. Since $X'$ is a Poincar\'e duality space, the morphism $-\cap[X']$ induces an isomorphism in homology with a degree shift of $n$ and therefore, the morphism $(\varphi_{\mathrm{P}})_0$ does as well.
\end{example}

\begin{definition}
An ANR homology $n$-manifold is a finite dimensional absolute neighborhood retract $X$ satisfying for every $x\in X$
$$\H_i(X,X\setminus\{x\})=\begin{cases}\Z &\text{if } i=n\\ 0 &\text{if } i\neq n. \end{cases}$$
\end{definition}

\begin{theorem} \label{existence and uniqueness for hom mnfds}
Let $X$ be a simply-connected regular pseudomanifold which is an $n$-dimensional Poincar\'e duality space with $n>4$.

For any homotopy equivalence between an ANR homology $n$-manifold and $X$, there exists a weak cobordism between a strong $n$-dimensional Poincar\'e comodule and $(\mathrm P,\varphi_{\mathrm{P}})$. Conversely, for every weak cobordism between a strong $n$-dimensional Poincar\'e comodule and $(\mathrm P,\varphi_{\mathrm{P}})$, there exists a homotopy equivalence between $X$ and an ANR homology $n$-manifold.

Two such homotopy equivalences are related by an $h$-cobordism relative to boundary if and only if their corresponding strong Poincar\'e comodules are related by a strong cobordism.
\end{theorem}

In order to obtain existence and uniqueness statements for topological manifold structures on $X$ one needs to ``tame the fundamental group" of the weak cobordisms involved.

\begin{definition}(Admissible cobordisms)
Using the notation of Definition \ref{cobordism definition}, a weak cobordism is said to be \textbf{admissible} if the the mapping cone of $(\phi/\!\varphi)_0$ is chain homotopy equivalent as $\S$-comodule to a sheaf-like comodule which equals $0$ in degrees less than or equal to $1$.
\end{definition}

\begin{theorem}\label{existence and uniqueness for top mnfds}
Let $X$ be a simply-connected regular pseudomanifold which is an $n$-dimensional Poincar\'e duality space with $n>4$.

For any homotopy equivalence between a topological $n$-manifold and $X$, there exists an admissible weak cobordism between a strong $n$-dimensional Poincar\'e comodule and $(\mathrm P,\varphi_{\mathrm{P}})$. Conversely, for every admissible weak cobordism between a strong $n$-dimensional Poincar\'e comodule and $(\mathrm P,\varphi_{\mathrm{P}})$, there exists a homotopy equivalence between $X$ and a topological $n$-manifold.

Two such homotopy equivalences are related by an $h$-cobordism relative to boundary if and only if their corresponding strong Poincar\'e comodules are related by a strong cobordism.
\end{theorem}

\begin{remark}
Starting with a homotopy equivalence in either of the theorems above, the weak cobordism obtained has the further property that the morphism from each of its boundary components induces an isomorphism in homology.

Also, in both theorems above, given a strong Poincar\'e complex weak cobordant to $(\mathrm P,\varphi_{\mathrm{P}})$, there exist another strong Poincar\'e complex, which is strong cobordant to the original one, and a weak cobordism between it and $(\mathrm P,\varphi_{\mathrm{P}})$ such that the morphism from each of its boundary components induces and isomorphism in homology.
\end{remark}

\begin{proof}[Proof of Theorem \ref{existence and uniqueness for hom mnfds} and Theorem \ref{existence and uniqueness for top mnfds}]
Both of the proofs are obtain by relating to the algebraic surgery theory as developed by Ranicki in \cite{Ran92}.

The differential graded category of connective sheaf-like comodules over $\chains(X)$ with duality $\T$ is, by Theorem \ref{sheaves fully-faithful to comodules}, equivalent to the category of connective complexes of projective sheaves over $X$ with Ranicki duality. This category is equivalent to the category of connective complexes of $X$-based modules defined in \cite[p.63]{Ran92}, see \cite[p.169]{RW90} for a proof, with the duality defined in \cite[p.75]{Ran92}.

The weak Poincar\'e comodule $(\mathrm P,\varphi_{\mathrm{P}})$ represents, under the above identification, the ``$(1/2)$-visible symmetric signature" of Ranicki, see Remark 16.8 \cite[p.181]{Ran92}. Also under this identification, weak cobordisms, admissible weak cobordism and strong cobordisms correspond to symmetric cobordisms in the algebraic bordism categories $\Lambda\langle 0\rangle(\Z,X)$, $\Lambda\langle 1/2\rangle(\Z,X)$ and $\Lambda\langle 0\rangle(\Z)_{*}(X)$ respectively, as defined in pages $157$, $164$ and $158$ of \cite{Ran92}.

The statements now follow from Theorem 17.4, Theorem 18.5, Proposition 25.7 and Remark 25.13 of \cite{Ran92}.

\end{proof}
%\end{document}