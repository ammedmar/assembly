% !TeX root = ../thesis.tex

\subsection{Conventions}

We write $\k$ for a commutative and unital ring, $(\Ch_\k, \ot, \k)$ for the symmetric monoidal category of chain complexes of free $\k$-modules, $\Sym_2$ for the group with only one non-identity element $T$, and $X$ for an unspecified simplicial complex.
In this work, the vertices of a simplicial complex are equipped with a partial order compatibly restricting to a total order on each simplex.

\subsection{Presheaves} \label{ss:presheaves}

Regard $X$ as a poset category with a morphism $x \to y$ whenever $x$ is a subsimplex of $y$.
The category of \textit{presheaves} on $X$ is the category of functors $\Fun(X^\op, \Ch_\k)$.
Given a presheaf $\cC$, we will denote the chain complex it associates to a simplex $x$ by $\cC_x$ and the linear map associated to a morphism $x \to y$ by $\cC_{x \to y}$.
We remark that this category is enriched over $\Ch_\k$.
These presheaves were termed ``local systems'' in \cite{ranicki1990assembly} where $X$ was allowed to be a semi-simplicial set.
We do not use this terminology since the associated topological sheaf --which we describe next-- need not be locally constant.

\subsection{Remark on topological sheaves} \label{ss:topological sheaf}

The set of simplices of $X$ can be endowed with a topology making presheaves as defined above into topological presheaves.
Explicitly, this topology $\tau$ on the set of simplices of $X$ is generated by subsets $\bar y = \set{x \in X : x \to y}$.
Notice that as poset categories $X$ and the category of open sets in $\tau$ are isomorphic.
Therefore, so are their categories of contravariant functors to $\Ch_\k$.
Furthermore, it can be verified that these topological presheaves are sheaves.

\subsection{Symmetric coalgebras}

We now introduce a derived notion of $\Sym_2$-equivariant coalgebras.
We avoid the term cocommutative since we do not consider higher arity compositions.
Let $W$ be the minimal free resolution of $\k$ by free $\k[\Sym_2]$-modules
\[
\k[\Sym_2]\set{e_0} \xla{1-T} \k[\Sym_2]\set{e_1} \xla{1+T} \k[\Sym_2]\set{e_2} \xla{1-T} \dotsb
\]
A \textit{symmetric coalgebra} is a chain complex $C$ with an $\Sym_2$-equivariant chain map
\[
W \ot C \to C \ot C
\]
where the action of $\Sym_2$ in the domain is induced from that in $W$ and in the target is given by transposition.
We will write $\Delta_i$ for $\Delta(e_i \ot -)$, noticing that
\[
\partial(\Delta_i) = \big( 1+(-1)^i T \big) \Delta_{i-1}
\]
with the convention $\Delta_{-1} = 0$.

\subsection{Remark on higher categories}
\label{ss:higher categories}

A strict infinity category is the limiting concept obtained from the recursive definition of an $n$-category as a category enriched in $(n-1)$-categories.
Building on ideas from \cite{brown1981cubes, kapranov1991polycategory, steiner2004omega}, in \cite{medina2020globular} a functor from a full subcategory of symmetric coalgebras to strict infinity categories was constructed and shown to be an equivalence onto a full subcategory recently characterized intrinsically in \cite{ozornova2022steiner}.
Additionally, this functor applied to the symmetric coalgebras of standard simplices defined by Steenrod and recalled in \cref{ss:cup-i}, yield Street's orientals \cite{street1987orientals} defining the nerve of infinity categories.

\subsection{Simplicial chains}

We denote the usual \textit{simplicial chains} of $X$ with coefficients in $\k$ by $\chains(X)$.
Explicitly,
\[
\chains(X)_m = \k\set[\big]{x \in X : \bars{x} = m},
\qquad \partial(x) = \textstyle\sum (-1)^u \face_u(x)
\]
where $\face_u(x)$ is the simplex obtained from $x$ by removing its vertex in position $u$.


\subsection{Alexander--Whitney coproduct} \label{ss:aw diagonal}

The \textit{Alexander--Whitney coproduct}
\[
\Delta_0 \colon \chains(X) \to \chains(X) \ot \chains(X)
\]
is defined on a basis element $[v_0, \dots, v_n]$ by
\begin{equation*} \label{e:alexander-whitney coalgebra}
	\Delta_0 \big( [v_0, \dots, v_n] \big) =
	\sum_{k=0}^n \ [v_0, \dots, v_k] \ot [v_k, \dots, v_n].
\end{equation*}
%and \textit{augmentation map}
%\[
%\aug \colon \chains(X) \to \k
%\]
%is defined on a basis element $x \in \chains(X)_n$ by
%\begin{equation*} \label{e:alexander-whitney coalgebra}
%	\Delta_0(x) = \sum_{u=0}^n \ \face_{u+1} \dotsm \face_{n}(x) \ot \face_{0} \dotsm \face_{u-1}(x).
%\end{equation*}
%\begin{equation*} \label{e:augmentation map}
%	\aug(x) =
%	\begin{cases}
%		1 & n = 0, \\
%		0 & n > 0.
%	\end{cases}
%\end{equation*}
%It satisfies
%\begin{gather}
%	\label{e:coassociativity relation}
%	(\Delta_0 \ot \, \id) \circ \Delta_0 = (\id \ot \Delta_0) \circ \Delta_0.
%	\label{e:counital relation}
%	(\aug \ot \, \id) \circ \Delta_0 = \id = (\id \ot \aug) \circ \Delta_0.
%\end{gather}
%making the (normalized) chains $\chains(X)$ of any simplicial set $X$ into a natural coassociative counital coalgebra, referred to as the \textit{Alexander--Whitney coalgebra} of $X$.
%We will use the following recursively defined notation:
%\begin{align*}
%	\Delta_0^1 &= \Delta_0, \\
%	\Delta_0^k &= (\Delta_0 \ot \, \id) \circ \Delta_0^{k-1},
%\end{align*}
%remarking that
%\[
%(\Delta_0 \ot \, \id) \circ \Delta_0^k = (\id \ot \Delta_0) \circ \Delta_0^k.
%\]

\subsection{The join product}

The \textit{join product}
\[
\ast \colon \chains(X) \ot \chains(X) \to \chains(X)
\]
is the natural degree~$1$ linear map defined on a basis element
\[
[v_0, \dots, v_p] \ot [v_{p+1}, \dots, v_q]
\]
to be $0$ if $\set{v_\ell : \ell = 0, \dots, q}$ is not the set of vertices of a simplex in $X$ or if $v_i = v_j$ for some $i \neq j$,
otherwise
\[
\ast \big(\left[v_0, \dots, v_p \right] \ot \left[v_{p+1}, \dots, v_q\right]\big) =
(-1)^{p} \sign(\pi) \left[v_{\pi(0)}, \dots, v_{\pi(q)}\right]
\]
where $\pi$ is the permutation that orders the vertices.
%It is an algebraic version of the usual join of faces in a simplex, please consult \cref{f:join of faces} for an example.

%\begin{figure}
%	\input{aux/join}
%	\caption{Geometric representation of the join product of two basis elements. It depicts the identity $\pr \big( [0] \otimes [1,2] \big) = [0,1,2]$.}
%	\label{f:join of faces}
%\end{figure}

\subsection{Steenrod construction} \label{ss:cup-i}

We will now describe Steenrod's construction of cup-$i$ coproducts \cite[p.293]{steenrod1947products} making $\chains(X)$ into a symmetric coalgebra.
Let
\[
\Delta \colon W \ot \chains(X) \to \chains(X) \ot \chains(X)
\]
be the $\Sym_2$-equivariant chain map determined by the linear maps $\Delta_i = \Delta(e_i \ot -)$ recursively defined for $i > 1$ by
\begin{equation*} \label{e:cup-i coproducts}
	\copr_i =
	(\ast \ot \id) \circ (\id \ot T\copr_{i-1}) \circ \copr_0.
\end{equation*}
We refer to \cite{real1996computability, gonzalez-diaz1999steenrod, mcclure2003multivariable, medina2021fast_sq} for alternative descriptions of cup-$i$ coproducts, all shown to be equivalent to Steenrod's original via an axiomatic characterization \cite{medina2022axiomatic}.
These cup-$i$ coproducts seem to be combinatorially fundamental, being constructed from the convex geometry of the standard simplex $\gsimplex^n$ in $\R^n$ given a generic orthonormal basis \cite{medina2022fib_poly}.

\subsection{Special cases}

We will use the following special values of the Steenrod cup-$i$ coalgebra, where $\pm$ stands for either $+$ or $-$.

\begin{lemma*}
	For a basis element $x$ of $\chains(X)_n$ we have that $\Delta_i(x) = 0$ if $i > n$, $\Delta_n(x) = \pm \, x \ot x$ and
	\[
	\Delta_{n-1}(x) \ =
	\sum_{u \text{ even}} \pm \face_u(x) \ot x \ +
	\sum_{u \text{ odd}} \pm \, x \ot \face_u(x).
	\]
\end{lemma*}
\begin{proof}
	The first statement follows immediately from the fact that $\Delta_i$ is of degree $i$.
	The next two are proven by induction on $n$.
	The base case can be easily verified.
	Let $x = [v_0, \dots, v_n]$ and assume the two claims for $n-1$.
	By definition, and the
	\begin{align*}
		\Delta_n(x) &=
		(\ast \ot \id) (\id \ot T \Delta_{n-1}) \Delta_0 (x) \\ &=
		(\ast \ot \id) (\id \ot T \Delta_{n-1}) \textstyle\sum [v_0, \dots v_i] \ot [v_i, \dots v_n] \\ &=
		(\ast \ot \id) (\id \ot T \Delta_{n-1}) \big( [v_0] \ot [v_0, \dots v_n] + [v_0, v_1] \ot [v_1, \dots v_n]\big)
	\end{align*}


\end{proof}

The specific signs could also be determined but we do not need this extra information.

\subsection{Remark on Steenrod squares}
\label{ss:steenrod squares}

We mention that Steenrod's introduction of the cup-$i$ coproducts $\Delta_i$ was motivated by the desire to construct finer invariants on the cohomology of spaces with mod 2 coefficients.
These are the celebrated \textit{Steenrod squares}
\[
Sq^k \colon H(X; \Ftwo) \to \colon H(X; \Ftwo)
\]
defined on a class $[\alpha]$ represented by a cocycle $\alpha$ of cohomological degree $n$ by
\[
Sq^k [\alpha] = \big[ (\alpha \ot \alpha) \Delta_{n-k}(-) \big].
\]

\subsection{Remark on $E_\infty$-structures}

Although we do not use the following fact in this work, we mention that the cup-$i$ coproducts of Steenrod are part of much larger structure derived from the diagonal of spaces; that of an $E_\infty$-coalgebra structure.
Explicitly, this structure is given by all maps $\chains(X) \to \chains(X)^{\ot r}$ for any $r > 0$ obtained from compositions of the Alexander--Whitney coproduct and the join product.
The underlying model for the $E_\infty$-operad use in this statement is presented in \cite{medina2020prop1, medina2021prop2}.
We mention that this $E_\infty$-coalgebra structure on simplicial chains is compatible with the $E_\infty$-coalgebras of McClure--Smith \cite{mcclure2003multivariable} and Berger--Fresse \cite{berger2004combinatorial}, and that analogues of the cup-$i$ coproducts for higher values of $r$ were defined in \cite{medina2021may_st} and implemented in \cite{medina2021comch}.

\subsection{Comodules}

A \textit{comodule} over a symmetric coalgebra $(C, \Delta)$ is a chain complex $M$ together with a chain map
\[
\nabla \colon W \ot M \to M \ot C.
\]
The examples of comodules that we will construct make the following diagram commute up to homotopy
\[
\begin{tikzcd}
	W \ot W \ot M \arrow[r, "\id \ot \nabla"] \arrow[d, "\id \ot \nabla"'] &
	W \ot M \ot C \arrow[d, "(\id \ot \Delta)(T \ot \id)"] \\
	W \ot M \ot C \arrow[r, "\nabla \ot \id"] &
	M \ot C \ot C,
\end{tikzcd}
\]
but we do need this additional property for the main theorem of this work (see \cref{ss:full embedding}).
We will denote by $\coMod_C^{\Sym_2}$ the category of such comodules with strict morphisms, i.e. linear maps $f \colon M \to M'$ making the following diagram
\[
\begin{tikzcd}
	W \ot M \arrow[d, "\nabla"'] \arrow[r, "\id\, \ot f"] &
	W \ot M' \arrow[d, "\nabla"] & \\
	M \ot C \arrow[r, "f \ot\, \id"] &
	M' \ot C
\end{tikzcd}
\]
commute.
We remark that this category is enriched in $\Ch_\k$.
We will write $\nabla_i$ for $\nabla(e_i \ot -)$ and $\nabla_i^T$ for $\nabla(Te_i \ot -)$.

\subsection{Closure cosheaf}

For any simplex $x$ in $X$ denote by $\bar x$ the subcomplex of $X$ containing the subsimplices of $x$.
The \textit{closure cosheaf} is the (covariant) functor from the poset category of $X$ to $\Ch_\k$ defined on objects by $x \mapsto \chains(\bar x)$ and on morphisms by canonical inclusions $\chains_{x \to y} \colon \chains(\bar x) \to \chains(\bar y)$.

For any simplex $y$ in $X$ and subsimplex $x$, we will find it convenient to identify the basis element $x$ in $\chains(\bar y)$ with the unique morphism $(x \to y)$, in particular $\chains_{y \to z} (x \to y) = (x \to z)$.
%, but we continue writing $y$ instead of $(y \to y)$.
%In particular, the canonical inclusion associated to $y \to z$ is given by
%\begin{align*}
%	\chains(\bar y) &\to \chains(\bar z) \\
%	(x \to y) &\mapsto (x \to z).
%\end{align*}

\subsection{Ranicki--Weiss Assembly} \label{ss:assembly}

The \textit{assembly functor}
\[
\assembly \colon \Fun(X^\op, \Ch_\k) \to \Ch_\k
\]
is given by tensoring over $X$ with the closure cosheaf.
Explicitly, let $\cC$ be a sheaf over $X$, then
\[
\assembly \cC = \bigoplus_{x \in X} \cC[x] \otimes \chains(\bar x) \ / \sim
\]
where for simplices $x \to y \to z$ and $c \in \cC[z]$ we have
\[
c \otimes (x \to z) \sim \cC_{y \to z}(c) \ot (x \to y)
\]
and for a morphism $F \colon \cC \to \cC'$ of presheaves over $X$ we have
\[
\assembly F[c \ot (x \to y)] = [F_y(c) \ot (x \to y)].
\]
We remark that this is a functor of categories enriched in $\Ch_\k$.

\subsection{Fundamental group}

TBW

\subsection{Lift to comodules}

We will describe a factorization of the assembly functor
\[
\assembly \colon \Fun(X^\op, \Ch_\k) \to \coMod^{\Sym_2}_{\chains(X;\k)} \to \Ch_{\k}
\]
where the second arrow is the forgetful functor.

Let $\cC$ be a presheaf over $X$ and let
\[
\nabla \colon W \ot \assembly \cC \to \assembly \cC \ot \chains(X)
\]
be the linear map defined for simplices $x \to y$ and $c \in \cC_y$ by
\[
\nabla_i \big( [c \ot (x \to y)] \big) =
\sum_{\lambda \in \Lambda} {[c \ot (x_\lambda^{(1)} \to y)]} \ot \alpha_\lambda \, x_\lambda^{(2)}
\]
and
\[
\nabla_i^T \big( [c \ot (x \to y)] \big) =
\sum_{\lambda \in \Lambda} {[c \ot (x_\lambda^{(2)} \to y)]} \ot \beta_\lambda \, x_\lambda^{(1)}
\]
where
\[
\Delta_i(x) = \sum_{\lambda \in \Lambda} \alpha_\lambda \ x_\lambda^{(1)} \ot x_\lambda^{(2)}
\]
and
$\beta_\lambda = \pm \alpha_\lambda$ with the sign of the transposition of $x_\lambda^{(1)}$ and $x_\lambda^{(2)}$.

\begin{lemma*}
	The map $\nabla$ naturally makes the assembly of a presheaf over $X$ into a comodule over the symmetric coalgebra $\chains(X)$.
\end{lemma*}

\begin{proof}
	The only part of the statement that is not immediate is that $\nabla_i$ is well defined for all $i \in \N$.
	Consider a presheaf $\cC$, simplices $x \to y \to z$ and an element $c \in \cC_z$, then
	\begin{align*}
		\nabla_i [c \ot (x \to z)] &=
		\sum {[c \ot (x^{(1)} \to z)]} \ot \alpha \, x^{(2)} \\ &=
		\sum {[\cC_{y \to z}(c) \ot (x^{(1)} \to y)]} \ot \alpha \, x^{(2)} \\ &=
		\nabla_i [\cC_{y \to z}(c) \ot (x \to y)]
	\end{align*}
	so $\nabla$ is well defined.
\end{proof}

\subsection{Full embedding} \label{ss:full embedding}

We now prove the main result of this work.
It states that the factorization of the assembly functor
\[
\assembly \colon \Fun(X^\op, \Ch_\k) \to \coMod^{\Sym_2}_{\chains(X;\k)} \to \Ch_{\k}
\]
is by a full and faithful functor followed by a forgetful one.

\begin{theorem*}
	The assembly functor from presheaves over $X$ to comodules over the symmetric coalgebra $\chains(X)$ is full and faithful.
\end{theorem*}

\begin{proof}
	% !TeX root = ../thesis.tex

Consider two presheaves $\cN$ and $\cN'$ over $X$ and the function
\[
\Fun(X^\op, \Ch)(\cN, \cN') \to \coMod^{\Sym_2}_{\chains(X)}(\assembly \cN, \assembly \cN').
\]
induced by the assembly functor.
We will show this to be a bijection.

For a simplex $x$ in $X$ we write throughout this proof $x$ for both the basis element it represents in $\chains(X)$ and in $\chains(\bar x)$ which corresponds to the identity $(x \to x)$.

We start by showing that the chain map above is injective.
Consider a morphism of presheaves $F \colon \cN \to \cN'$ with $\assembly F = 0$.
For each simplex $x$ choose a basis $B_x$ of $\cN_x$ and notice that
\[
B = \set[\big]{[x \ot b] \mid x \in X, b \in B_x}
\]
is a basis for $\assembly \cN$.
By assumption we have $\assembly F [x \ot b] = [x \ot F_x(b)] = 0$ for each $[x \ot b] \in B$, which implies $F_x(b) = 0$ and consequently $F = 0$.

Let us now move to surjectivity.
Consider a morphism of comodules $f \colon \assembly \cN \to \assembly \cN'$.
We will construct a morphisms f presheaves $F \colon \cN \to \cN'$ such that $\assembly F = f$.
Consider $x \in X$, $c \in \cN[x]$ and the class $[x \ot c] \in \assembly \cN$.
Its image under $f$ is of the form
\[
f[x \ot c] = \sum_{\lambda \in \Lambda} \, [x_\lambda \ot c_\lambda]
\]
where $x_\lambda \in X$ and $c_\lambda \in \cN'[x_\lambda]$.
We will first show that for each $\lambda$ in the (finite) sum above $\bars{x_\lambda} \leq \bars{x}$.
Let $i = \max \set[\big]{\bars{x_\lambda} : \lambda \in \Lambda}$ and $\Lambda_i = \set[\big]{\lambda \in \Lambda : \bars{x_\lambda} = i}$.
Notice that $\Delta_i(x_{\lambda}) = x_{\lambda} \ot x_{\lambda}$ for $\lambda \in \Lambda_i$ and $\Delta_i(x_{\lambda}) = 0$ for $\lambda \notin \Lambda_i$.
Assume $i > \bars{x}$ so $\Delta_i(x) = 0$.
%and considering $\nabla_i$.
%The comodule map property of $f$ implies, since , that
Therefore,
\begin{align*}
	0 =
	(\id \ot f) \circ \nabla_i [x \ot c] =
	\nabla_i \sum_{\lambda \in \Lambda} \, [x_\lambda \ot c_\lambda] =
	\sum_{\lambda \in \Lambda_i} \, x_\lambda \ot [x_\lambda \ot c_\lambda],
\end{align*}
which implies $[x_\lambda \ot c_\lambda] = 0$ and consequently $c_{\lambda} = 0$ for each $\lambda \in \Lambda_i$.

We will now show that $\Lambda$ has a single element $\lambda$ with $x_\lambda = x$.
Let $i = \bars{x}$, so
\begin{equation*} % \label{e:first in surjectivity}
	(\id \ot f) \circ \nabla_i \big( [x \ot c] \big) =
	x \ot f \big( [x \ot c] \big) =
	\sum_{\lambda \in \Lambda} \, x \ot [x_\lambda \ot c_\lambda]
\end{equation*}
and
\begin{equation*} % \label{e:second in surjectivity}
	\nabla_i \circ f \big( [x \ot c] \big) =
	\sum_{\lambda \in \Lambda_i} \, x_\lambda \ot [x_\lambda \ot c_\lambda].
\end{equation*}
Consequently
\[
\sum_{\lambda \in \Lambda} \, x \ot [x_\lambda \ot c_\lambda] =
\sum_{\lambda \in \Lambda_i} \, x_\lambda \ot [x_\lambda \ot c_\lambda],
\]
from which the claim follows.
%The comodule property of $f$ implies the elements in \cref{e:first in surjectivity,e:second in surjectivity} are equal, from which it follows that $\Lambda$ contains a single element $\lambda$ and $x_{\lambda} = x$.
Denoting $c_\lambda$ by $c'$ we have that $f \big( [x \ot c] \big) = [x \ot c']$, which we use to define $F_x(c) = c'$.

We will next verify that $F$ is a well defined morphism of presheaves.
That is to say, that for any morphism $x \to y$ and $c \in \cN_y$ we have
\[
\cN'_{x \to y} \circ F_y(c) = F_x \circ \cN_{x \to y}(c).
\]
It suffices to consider morphisms $d_u x \to x$ where $d_u x$ is obtained by removing the vertex of $x$ in position $u$.
Let $i = \bars{x}-1$ so that
\[
\Delta_i(x) \ =
\sum_{u \text{ even}} \pm \face_u(x) \ot x \ +
\sum_{u \text{ odd}} \pm \, x \ot \face_u(x).
\]
Therefore, on one hand we have that $(\id \ot f) \circ \nabla_i [x \ot c]$ is equal to
\begin{align*}
%	&\sum_{u \text{ even}} \pm f [c \ot \face_u(x) \to x] \ot x \ +
%	\sum_{u \text{ odd}} \pm \, f [c \ot x] \ot \face_u(x) \\ =
%	&\sum_{u \text{ even}} \pm f [\cN_{\face_u(x) \to x}(c) \ot \face_u(x)] \ot x \ +
%	\sum_{u \text{ odd}} \pm \, f [c \ot x] \ot \face_u(x) \\ =
	&\sum_{u \text{ even}} \pm x \ot [\face_u(x) \ot \cN_{\face_u(x) \to x}(c)'] \ +
	\sum_{u \text{ odd}} \pm \, \face_u(x) \ot [x \ot c'],
\end{align*}
while on the other that $\nabla_i \circ f [x \ot c] = \nabla_i [x \ot c']$ is equal to
\begin{align*}
%	&\sum_{u \text{ even}} \pm [c' \ot \face_u(x) \to x] \ot x \ +
%	\sum_{u \text{ odd}} \pm \, [c' \ot x] \ot \face_u(x) \\
	&\sum_{u \text{ even}} \pm x \ot [\face_u(x) \ot \cN'_{\face_u(x) \to x}(c')] \ +
	\sum_{u \text{ odd}} \pm \, \face_u(x) \ot [x \ot c'].
\end{align*}
Putting these together it follows that
\[
	\sum_{u \text{ even}} \pm [\face_u(x) \ot \cN_{\face_u(x) \to x}(c)'] \ =
	\sum_{u \text{ even}} \pm [\face_u(x) \ot \cN'_{\face_u(x) \to x}(c')]
\]
and, consequently, that for $u$ even
\begin{align*}
	F_{\face_u(x)} \circ \cN_{\face_u(x) \to x}(c) &=
	\cN_{\face_u(x) \to x}(c)' \\ &=
	\cN'_{\face_u(x) \to x}(c') \\ &=
	\cN'_{\face_u(x) \to x} \circ F_{x}(c)
\end{align*}
as desired.
For $u$ odd we repeat the same argument using $\nabla_i^T$ instead of $\nabla_i$.
%\begin{equation*} % \label{e:first in surjectivity}
%	(f \ot \id) \circ \nabla_i \big( [c \ot x] \big) =
%	f \big( [c \ot x] \big) \ot x =
%	\sum_{\lambda \in \Lambda} \, [c_\lambda \ot x_\lambda] \ot x
%\end{equation*}
%and
%\begin{equation*} % \label{e:second in surjectivity}
%	\nabla_i \circ f \big( [c \ot x] \big) =
%	\sum_{\lambda \in \Lambda_i} \, [c_\lambda \ot x_\lambda] \ot x_\lambda.
%\end{equation*}
%Consequently
%\[
%\sum_{\lambda \in \Lambda} \, [c_\lambda \ot x_\lambda] \ot x =
%\sum_{\lambda \in \Lambda_i} \, [c_\lambda \ot x_\lambda] \ot x_\lambda,
%\]
\end{proof}

\subsection{Conclusions}

TBW