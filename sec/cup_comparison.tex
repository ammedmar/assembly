\documentclass{amsart}
\usepackage{microtype}
\usepackage{amssymb}
\usepackage{mathtools}
\usepackage{tikz-cd}
\usepackage{mathbbol} % changes \mathbb{} and adds more support
\input{../aux/usualcmds}
\begin{document}

	\begin{gather*}
		\Delta_{i+1}(x) = \\
		(\ast \ot \id)(\id \ot T\Delta_i) \Delta_0(x) = \\
		(\ast \ot \id) \textstyle\sum_k (\id \ot T\Delta_i) d_{k+1} \dotsm d_n(x) \ot d_{0} \dotsm d_{k-1}(x) = \\
		(\ast \ot \id)\textstyle\sum_k \sum_{U \in \rP^{n-k}_{n-k-i}} (-1)^{(n-k)i+\bars{U^0}\bars{U^1}} d_{k+1} \dotsm d_n(x) \ot d_{U^1} d_{0} \dotsm d_{k-1}(x) \ot d_{U^0} d_{0} \dotsm d_{k-1}(x) = \\
		\textstyle\sum_k \sum_{U \in \rP^{n-k}_{n-k-i}} (-1)^{k+(n-k)i+\bars{U^0}\bars{U^1}} \big( d_{k+1} \dotsm d_n(x) \, \ast d_{U^1} d_{0} \dotsm d_{k-1}(x) \big) \ot d_{U^0} d_{0} \dotsm d_{k-1}(x) = \\
		\textstyle\sum_k \sum_{U \in \rP^{n-k}_{n-k-i}} (-1)^{k+(n-k)i+\bars{U^0}\bars{U^1}} \big( d_{k+1} \dotsm d_n(x) \, \ast d_{0} \dotsm d_{k-1} d_{U+k^1}(x) \big) \ot d_{0} \dotsm d_{k-1} d_{U+k^0}(x)
	\end{gather*}
	where in the last equality we used the simplicial identity and the notation
	\[
	U^\varepsilon \pm k = \{u \pm k \mid u \in U^\varepsilon\}.
	\]
	Since ...
	\begin{align*}
		\Delta_{i+1}(x) &=
		\sum_k \sum_{U \in \rP^{n-k}_{n-k-i} \,|\, 0 \in U} d_{k+(U^1 \setminus 0)}(x) \ot d_{0} \dotsm d_{u-1} d_{k+U^0}(x).
	\end{align*}
	Now consider
	\[
	\sum_{V \in \rP^{n}_{n-i-1}} d_{V^0}(x) \ot d_{V^1}(x).
	\]
	We partition $\rP^{n}_{n-i-1}$ using the length of the maximal consecutive subset $\{0,\dots,k-1\}$ denoting the corresponding part $\rP^{n,k}_{n-i-1}$.
	Take $V \in \rP^{n,k}_{n-i-1}$ which implies that $V^1 = \{0,\dots,k-1\} \cup \widetilde V^1$ for some $\widetilde V^1$ and $k \notin V$.
	Let $U = (V^0 \cup \widetilde V^1 \cup \{k\})-k$.
	This is an element in $\rP^{n-k}_{n-k-i}$ with $0 \in U$ and notice that
	\[
	U^0 = \{0\} \cup (\widetilde V_1-k),
	\qquad
	U^1 = V^0-k.
	\]
	Therefore,
	$V_0 = (U^1 \setminus \{0\})+k
	\qquad
	V^1 = \{0,\dots,k-1\} \cup V^0$.

	\newpage
	mod 2
	\begin{align*}
		\Delta_{i+1}(x) &=
		(\ast \ot \id)(\id \ot T\Delta_i) \Delta_0(x) \\ &=
		(\ast \ot \id)(\id \ot T\Delta_i) \textstyle\sum_u d_{u+1} \dotsm d_n(x) \ot d_{0} \dotsm d_{u-1}(x) \\ &=
		\textstyle\sum_u \sum_{U \in \rP^{n-u}_{n-u-i}} \big( d_{u+1} \dotsm d_n(x) \, \ast d_{U^1} d_{0} \dotsm d_{u-1}(x) \big) \ot d_{U^0} d_{0} \dotsm d_{u-1}(x) \\ &=
		\textstyle\sum_u \sum_{U \in \rP^{n-u}_{n-u-i}} \big( d_{u+1} \dotsm d_n(x) \, \ast d_{0} \dotsm d_{u-1} d_{u+U^1}(x) \big) \ot d_{0} \dotsm d_{u-1} d_{u+U^0}(x)
	\end{align*}
	where in the last equality we used the simplicial identity and the notation
	\[
	u+U^\varepsilon = \{u+w \mid w\in U^\varepsilon\}.
	\]
	Since ...
	\begin{align*}
		\Delta_{i+1}(x) &=
		\sum_u \sum_{U \in \rP^{n-u}_{n-u-i} \,|\, 0 \in U} d_{u+(U^1 \setminus 0)}(x) \ot d_{0} \dotsm d_{u-1} d_{u+U^0}(x).
	\end{align*}
	Now consider
	\[
	\sum_{V \in \rP^{n}_{n-i-1}} d_{V^0}(x) \ot d_{V^1}(x).
	\]
	We partition $\rP^{n}_{n-i-1}$ using the length of the maximal consecutive subset $\{0,\dots,k-1\}$ denoting the corresponding part $\rP^{n,k}_{n-i-1}$.
	Take $V \in \rP^{n,k}_{n-i-1}$ which implies that $V^1 = \{0,\dots,k-1\} \cup \widetilde V^1$ for some $\widetilde V^1$ and $k \notin V$.
	Let $U = (V^0 \cup \widetilde V^1 \cup \{k\})-k$.
	This is an element in $\rP^{n-k}_{n-k-i}$ with $0 \in U$ and notice that
	\[
	U^0 = \{0\} \cup (\widetilde V_1-k),
	\qquad
	U^1 = V^0-k.
	\]
	Therefore,
	$V_0 = (U^1 \setminus \{0\})+k
	\qquad
	V^1 = \{0,\dots,k-1\} \cup V^0$.
\end{document}