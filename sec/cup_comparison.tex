\documentclass{amsart}
\usepackage{microtype}
\usepackage{amssymb}
\usepackage{mathtools}
\usepackage{tikz-cd}
\usepackage{mathbbol} % changes \mathbb{} and adds more support
% environments
\newtheorem{theorem}{Theorem}
\newtheorem*{theorem*}{Theorem}
\newtheorem{proposition}[theorem]{Proposition}
\newtheorem*{proposition*}{Proposition}
\newtheorem{lemma}[theorem]{Lemma}
\newtheorem*{lemma*}{Lemma}
\newtheorem{corollary}[theorem]{Corollary}
\newtheorem*{corollary*}{Corollary}

\theoremstyle{definition}
\newtheorem{definition}[theorem]{Definition}
\newtheorem*{definition*}{Definition}
\newtheorem{remark}[theorem]{Remark}
\newtheorem*{remark*}{Remark}
\newtheorem{example}[theorem]{Example}
\newtheorem*{example*}{Example}
\newtheorem{construction}[theorem]{Construction}
\newtheorem*{construction*}{Construction}
\newtheorem{convention}[theorem]{Convention}
\newtheorem*{convention*}{Convention}
\newtheorem{terminology}[theorem]{Terminology}
\newtheorem*{terminology*}{Terminology}
\newtheorem{notation}[theorem]{Notation}
\newtheorem*{notation*}{Notation}
\newtheorem{question}[theorem]{Question}
\newtheorem*{question*}{Question}

% hyphenation
\hyphenation{co-chain}
\hyphenation{co-chains}
\hyphenation{co-al-ge-bra}
\hyphenation{co-al-ge-bras}
\hyphenation{co-bound-ary}
\hyphenation{co-bound-aries}
\hyphenation{Func-to-rial-i-ty}
\hyphenation{colim-it}
\hyphenation{di-men-sional}

% basics
\DeclareMathOperator{\face}{d}
\DeclareMathOperator{\dege}{s}
\DeclareMathOperator{\bd}{\partial}
\DeclareMathOperator{\sign}{sign}
\newcommand{\ot}{\otimes}
\DeclareMathOperator{\EZ}{EZ}
\DeclareMathOperator{\AW}{AW}
\newcommand{\diag}{\mathrm{D}}


% sets and spaces
\newcommand{\N}{\mathbb{N}}
\newcommand{\Z}{\mathbb{Z}}
\newcommand{\Q}{\mathbb{Q}}
\newcommand{\R}{\mathbb{R}}
\renewcommand{\k}{\Bbbk}
\newcommand{\Sym}{\mathbb{S}}
\newcommand{\Cyc}{\mathbb{C}}
\newcommand{\Ftwo}{{\mathbb{F}_2}}
\newcommand{\Fp}{{\mathbb{F}_p}}
\newcommand{\Cp}{{\mathbb{C}_p}}
\newcommand{\gsimplex}{\mathbb{\Delta}}
\newcommand{\gcube}{\mathbb{I}}

% categories
\newcommand{\Cat}{\mathsf{Cat}}
\newcommand{\Fun}{\mathsf{Fun}}
\newcommand{\Set}{\mathsf{Set}}
\newcommand{\Top}{\mathsf{Top}}
\newcommand{\CW}{\mathsf{CW}}
\newcommand{\Ch}{\mathsf{Ch}}
\newcommand{\simplex}{\triangle}
\newcommand{\sSet}{\mathsf{sSet}}
\newcommand{\cube}{\square}
\newcommand{\cSet}{\mathsf{cSet}}
\newcommand{\Alg}{\mathsf{Alg}}
\newcommand{\coAlg}{\mathsf{coAlg}}
\newcommand{\biAlg}{\mathsf{biAlg}}
\newcommand{\sGrp}{\mathsf{sGrp}}
\newcommand{\Mon}{\mathsf{Mon}}
\newcommand{\SymMod}{\mathsf{Mod}_{\Sym}}
\newcommand{\SymBimod}{\mathsf{biMod}_{\Sym}}
\newcommand{\operads}{\mathsf{Oper}}
\newcommand{\props}{\mathsf{Prop}}

% operators
\DeclareMathOperator{\free}{F}
\DeclareMathOperator{\forget}{U}
\DeclareMathOperator{\yoneda}{\mathcal{Y}}
\DeclareMathOperator{\Sing}{Sing}
\newcommand{\loops}{\Omega}
\DeclareMathOperator{\cobar}{\mathbf{\Omega}}
\DeclareMathOperator{\proj}{\pi}
\DeclareMathOperator{\incl}{\iota}

% chains
%\DeclareMathOperator{\chains}{N}
%\DeclareMathOperator{\cochains}{N^{\vee}}
%\DeclareMathOperator{\gchains}{C}

% pair delimiters (mathtools)
\DeclarePairedDelimiter\bars{\lvert}{\rvert}
\DeclarePairedDelimiter\norm{\lVert}{\rVert}
\DeclarePairedDelimiter\angles{\langle}{\rangle}
\DeclarePairedDelimiter\set{\{}{\}}

% other
\newcommand{\id}{\mathsf{id}}
\renewcommand{\th}{\mathrm{th}}
\newcommand{\op}{\mathrm{op}}
\DeclareMathOperator*{\colim}{colim}
\DeclareMathOperator{\coker}{coker}
\DeclareMathOperator{\Med}{\mathcal{M}}
\newcommand{\Hom}{\mathrm{Hom}}
\newcommand{\End}{\mathrm{End}}
\newcommand{\coEnd}{\mathrm{coEnd}}
\newcommand{\biEnd}{\mathrm{biEnd}}
\newcommand{\xla}[1]{\xleftarrow{#1}}
\newcommand{\xra}[1]{\xrightarrow{#1}}
\newcommand{\defeq}{\stackrel{\mathrm{def}}{=}}

% letters
\newcommand{\bC}{\mathbb{C}}
\newcommand{\bk}{\mathbb{k}}
\newcommand{\bF}{\mathbb{F}}

\newcommand{\sA}{\mathsf{A}}
\newcommand{\sB}{\mathsf{B}}
\newcommand{\sC}{\mathsf{C}}

\newcommand{\cA}{\mathcal{A}}
\newcommand{\cB}{\mathcal{B}}
\newcommand{\cC}{\mathcal{C}}
\newcommand{\cD}{\mathcal{D}}
\newcommand{\cE}{\mathcal{E}}
\newcommand{\cH}{\mathcal{H}}
\newcommand{\cL}{\mathcal{L}}
\newcommand{\cM}{\mathcal{M}}
\newcommand{\cN}{\mathcal{N}}
\newcommand{\cO}{\mathcal{O}}
\newcommand{\cP}{\mathcal{P}}
\newcommand{\cW}{\mathcal{W}}
\newcommand{\cX}{\mathcal{X}}
\newcommand{\cY}{\mathcal{Y}}
\newcommand{\cZ}{\mathcal{Z}}

\newcommand{\rA}{\mathrm{A}}
\newcommand{\rB}{\mathrm{B}}
\newcommand{\rC}{\mathrm{C}}
\newcommand{\rD}{\mathrm{D}}
\newcommand{\rH}{\mathrm{H}}
\newcommand{\rK}{\mathrm{K}}
\newcommand{\rM}{\mathrm{M}}
\newcommand{\rP}{\mathrm{P}}
\newcommand{\rT}{\mathrm{T}}
\newcommand{\rW}{\mathrm{W}}

% comments
\newcommand{\anibal}[1]{\textcolor{blue}{\underline{Anibal}: #1}}

\begin{document}

	\begin{gather*}
		\Delta_{i+1}(x) = \\
		(\ast \ot \id)(\id \ot T\Delta_i) \Delta_0(x) = \\
		(\ast \ot \id) \textstyle\sum_k (\id \ot T\Delta_i) d_{k+1} \dotsm d_n(x) \ot d_{0} \dotsm d_{k-1}(x) = \\
		(\ast \ot \id)\textstyle\sum_k \sum_{U \in \rP^{n-k}_{n-k-i}} (-1)^{(n-k)i+\bars{U^0}\bars{U^1}} d_{k+1} \dotsm d_n(x) \ot d_{U^1} d_{0} \dotsm d_{k-1}(x) \ot d_{U^0} d_{0} \dotsm d_{k-1}(x) = \\
		\textstyle\sum_k \sum_{U \in \rP^{n-k}_{n-k-i}} (-1)^{k+(n-k)i+\bars{U^0}\bars{U^1}} \big( d_{k+1} \dotsm d_n(x) \, \ast d_{U^1} d_{0} \dotsm d_{k-1}(x) \big) \ot d_{U^0} d_{0} \dotsm d_{k-1}(x) = \\
		\textstyle\sum_k \sum_{U \in \rP^{n-k}_{n-k-i}} (-1)^{k+(n-k)i+\bars{U^0}\bars{U^1}} \big( d_{k+1} \dotsm d_n(x) \, \ast d_{0} \dotsm d_{k-1} d_{U+k^1}(x) \big) \ot d_{0} \dotsm d_{k-1} d_{U+k^0}(x)
	\end{gather*}
	where in the last equality we used the simplicial identity and the notation
	\[
	U^\varepsilon \pm k = \{u \pm k \mid u \in U^\varepsilon\}.
	\]
	Since ...
	\begin{align*}
		\Delta_{i+1}(x) &=
		\sum_k \sum_{U \in \rP^{n-k}_{n-k-i} \,|\, 0 \in U} d_{k+(U^1 \setminus 0)}(x) \ot d_{0} \dotsm d_{u-1} d_{k+U^0}(x).
	\end{align*}
	Now consider
	\[
	\sum_{V \in \rP^{n}_{n-i-1}} d_{V^0}(x) \ot d_{V^1}(x).
	\]
	We partition $\rP^{n}_{n-i-1}$ using the length of the maximal consecutive subset $\{0,\dots,k-1\}$ denoting the corresponding part $\rP^{n,k}_{n-i-1}$.
	Take $V \in \rP^{n,k}_{n-i-1}$ which implies that $V^1 = \{0,\dots,k-1\} \cup \widetilde V^1$ for some $\widetilde V^1$ and $k \notin V$.
	Let $U = (V^0 \cup \widetilde V^1 \cup \{k\})-k$.
	This is an element in $\rP^{n-k}_{n-k-i}$ with $0 \in U$ and notice that
	\[
	U^0 = \{0\} \cup (\widetilde V_1-k),
	\qquad
	U^1 = V^0-k.
	\]
	Therefore,
	$V_0 = (U^1 \setminus \{0\})+k
	\qquad
	V^1 = \{0,\dots,k-1\} \cup V^0$.

	\newpage
	mod 2
	\begin{align*}
		\Delta_{i+1}(x) &=
		(\ast \ot \id)(\id \ot T\Delta_i) \Delta_0(x) \\ &=
		(\ast \ot \id)(\id \ot T\Delta_i) \textstyle\sum_u d_{u+1} \dotsm d_n(x) \ot d_{0} \dotsm d_{u-1}(x) \\ &=
		\textstyle\sum_u \sum_{U \in \rP^{n-u}_{n-u-i}} \big( d_{u+1} \dotsm d_n(x) \, \ast d_{U^1} d_{0} \dotsm d_{u-1}(x) \big) \ot d_{U^0} d_{0} \dotsm d_{u-1}(x) \\ &=
		\textstyle\sum_u \sum_{U \in \rP^{n-u}_{n-u-i}} \big( d_{u+1} \dotsm d_n(x) \, \ast d_{0} \dotsm d_{u-1} d_{u+U^1}(x) \big) \ot d_{0} \dotsm d_{u-1} d_{u+U^0}(x)
	\end{align*}
	where in the last equality we used the simplicial identity and the notation
	\[
	u+U^\varepsilon = \{u+w \mid w\in U^\varepsilon\}.
	\]
	Since ...
	\begin{align*}
		\Delta_{i+1}(x) &=
		\sum_u \sum_{U \in \rP^{n-u}_{n-u-i} \,|\, 0 \in U} d_{u+(U^1 \setminus 0)}(x) \ot d_{0} \dotsm d_{u-1} d_{u+U^0}(x).
	\end{align*}
	Now consider
	\[
	\sum_{V \in \rP^{n}_{n-i-1}} d_{V^0}(x) \ot d_{V^1}(x).
	\]
	We partition $\rP^{n}_{n-i-1}$ using the length of the maximal consecutive subset $\{0,\dots,k-1\}$ denoting the corresponding part $\rP^{n,k}_{n-i-1}$.
	Take $V \in \rP^{n,k}_{n-i-1}$ which implies that $V^1 = \{0,\dots,k-1\} \cup \widetilde V^1$ for some $\widetilde V^1$ and $k \notin V$.
	Let $U = (V^0 \cup \widetilde V^1 \cup \{k\})-k$.
	This is an element in $\rP^{n-k}_{n-k-i}$ with $0 \in U$ and notice that
	\[
	U^0 = \{0\} \cup (\widetilde V_1-k),
	\qquad
	U^1 = V^0-k.
	\]
	Therefore,
	$V_0 = (U^1 \setminus \{0\})+k
	\qquad
	V^1 = \{0,\dots,k-1\} \cup V^0$.
\end{document}