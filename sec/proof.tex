% !TeX root = ../thesis.tex

Consider two presheaves $\cC$ and $\cC'$ over $X$ and the function
\[
\Fun(X^\op, \Ch_\k)(\cC, \cC') \to \coMod^{\Sym_2}_{\chains(X;\, \k)}(\assembly \cC, \assembly \cC').
\]
induced by the assembly functor.
We will show this to be a bijection.

For a simplex $x$ in $X$ we write throughout this proof $x$ for both the basis element it represents in $\chains(X)$ and in $\chains(\bar x)$ which corresponds to the identity $(x \to x)$.

We start by showing that the chain map above is injective.
Consider a morphism of presheaves $F \colon \cC \to \cC'$ with $\assembly F = 0$.
For each simplex $x$ choose a basis $B_x$ of $\cC_x$ and notice that
\[
B = \set[\big]{[b \ot x] \mid x \in X, b \in B_x}
\]
is a basis for $\assembly \cC$.
By assumption we have $\assembly F [b \ot x] = [F_x(b) \ot x] = 0$ for each $[b \ot x] \in B$, which implies $F_x(b) = 0$ and consequently $F = 0$.

Let us now move to surjectivity.
Consider a morphism of comodules $f \colon \assembly \cC \to \assembly \cC'$.
We will construct a morphisms f presheaves $F \colon \cC \to \cC'$ such that $\assembly F = f$.
Consider $x \in X$, $c \in \cC[x]$ and the class $[c \ot x] \in \assembly \cC$.
Its image under $f$ is of the form
\[
f[c \ot x] = \sum_{\lambda \in \Lambda} \, [c_\lambda \ot x_\lambda]
\]
where $x_\lambda \in X$ and $c_\lambda \in \cC'[x_\lambda]$.
We will first show that for each $\lambda$ in the (finite) sum above $\bars{x_\lambda} \leq \bars{x}$.
Let $i = \max \set[\big]{\bars{x_\lambda} : \lambda \in \Lambda}$ and $\Lambda_i = \set[\big]{\lambda \in \Lambda : \bars{x_\lambda} = i}$.
Notice that $\Delta_i(x_{\lambda}) = x_{\lambda} \ot x_{\lambda}$ for $\lambda \in \Lambda_i$ and $\Delta_i(x_{\lambda}) = 0$ for $\lambda \notin \Lambda_i$.
Assume $i > \bars{x}$ so $\Delta_i(x) = 0$.
%and considering $\nabla_i$.
%The comodule map property of $f$ implies, since , that
Therefore,
\begin{align*}
	0 =
	(f \ot \id) \circ \nabla_i [c \ot x] =
	\nabla_i \sum_{\lambda \in \Lambda} \, [c_\lambda \ot x_\lambda] =
	\sum_{\lambda \in \Lambda_i} \, [c_{\lambda} \ot x_{\lambda}] \ot x_{\lambda},
\end{align*}
which implies $[c_{\lambda} \ot x_{\lambda}] = 0$ and consequently $c_{\lambda} = 0$ for each $\lambda \in \Lambda_i$.

We will now show that $\Lambda$ has a single element $\lambda$ with $x_\lambda = x$.
Let $i = \bars{x}$, so
\begin{equation*} % \label{e:first in surjectivity}
	(f \ot \id) \circ \nabla_i \big( [c \ot x] \big) =
	f \big( [c \ot x] \big) \ot x =
	\sum_{\lambda \in \Lambda} \, [c_\lambda \ot x_\lambda] \ot x
\end{equation*}
and
\begin{equation*} % \label{e:second in surjectivity}
	\nabla_i \circ f \big( [c \ot x] \big) =
	\sum_{\lambda \in \Lambda_i} \, [c_\lambda \ot x_\lambda] \ot x_\lambda.
\end{equation*}
Consequently
\[
\sum_{\lambda \in \Lambda} \, [c_\lambda \ot x_\lambda] \ot x =
\sum_{\lambda \in \Lambda_i} \, [c_\lambda \ot x_\lambda] \ot x_\lambda,
\]
from which the claim follows.
%The comodule property of $f$ implies the elements in \cref{e:first in surjectivity,e:second in surjectivity} are equal, from which it follows that $\Lambda$ contains a single element $\lambda$ and $x_{\lambda} = x$.
Denoting $c_\lambda$ by $c'$ we have that $f \big( [c \ot x] \big) = [c' \ot x]$, which we use to define $F_x(c) = c'$.

We will next verify that $F$ is a well defined morphism of presheaves.
That is to say, that for any morphism $x \to y$ and $c \in \cC_y$ we have
\[
\cC'_{x \to y} \circ F_y(c) = F_x \circ \cC_{x \to y}(c).
\]
It suffices to consider morphisms $d_u x \to x$ where $d_u x$ is obtained by removing the vertex of $x$ in position $u$.
Let $i = \bars{x}-1$ so that
\[
\Delta_i(x) \ =
\sum_{u \text{ even}} \pm \face_u(x) \ot x \ +
\sum_{u \text{ odd}} \pm \, x \ot \face_u(x).
\]
Therefore, on one hand we have that $(f \ot \id) \circ \nabla_i [c \ot x]$ is equal to
\begin{align*}
%	&\sum_{u \text{ even}} \pm f [c \ot \face_u(x) \to x] \ot x \ +
%	\sum_{u \text{ odd}} \pm \, f [c \ot x] \ot \face_u(x) \\ =
%	&\sum_{u \text{ even}} \pm f [\cC_{\face_u(x) \to x}(c) \ot \face_u(x)] \ot x \ +
%	\sum_{u \text{ odd}} \pm \, f [c \ot x] \ot \face_u(x) \\ =
	&\sum_{u \text{ even}} \pm [\cC_{\face_u(x) \to x}(c)' \ot \face_u(x)] \ot x \ +
	\sum_{u \text{ odd}} \pm \, [c' \ot x] \ot \face_u(x),
\end{align*}
while the other that $\nabla_i \circ f [c \ot x] = \nabla_i [c' \ot x]$ is equal to
\begin{align*}
%	&\sum_{u \text{ even}} \pm [c' \ot \face_u(x) \to x] \ot x \ +
%	\sum_{u \text{ odd}} \pm \, [c' \ot x] \ot \face_u(x) \\
	&\sum_{u \text{ even}} \pm [\cC'_{\face_u(x) \to x}(c') \ot \face_u(x)] \ot x \ +
	\sum_{u \text{ odd}} \pm \, [c' \ot x] \ot \face_u(x).
\end{align*}
Putting these together it follows that
\[
\sum_{u \text{ even}} \pm [\cC_{\face_u(x) \to x}(c)' \ot \face_u(x)] \ =
\sum_{u \text{ even}} \pm [\cC'_{\face_u(x) \to x}(c') \ot \face_u(x)]
\]
and, consequently, that for $u$ even
\begin{align*}
	F_{\face_u(x)} \circ \cC_{\face_u(x) \to x}(c) &=
	\cC_{\face_u(x) \to x}(c)' =
	\cC'_{\face_u(x) \to x}(c') \\ &=
	\cC'_{\face_u(x) \to x} \circ F_{x}(c)
\end{align*}
as desired.
For $u$ odd we repeat the same argument using $\nabla_i^T$ instead of $\nabla_i$.
%\begin{equation*} % \label{e:first in surjectivity}
%	(f \ot \id) \circ \nabla_i \big( [c \ot x] \big) =
%	f \big( [c \ot x] \big) \ot x =
%	\sum_{\lambda \in \Lambda} \, [c_\lambda \ot x_\lambda] \ot x
%\end{equation*}
%and
%\begin{equation*} % \label{e:second in surjectivity}
%	\nabla_i \circ f \big( [c \ot x] \big) =
%	\sum_{\lambda \in \Lambda_i} \, [c_\lambda \ot x_\lambda] \ot x_\lambda.
%\end{equation*}
%Consequently
%\[
%\sum_{\lambda \in \Lambda} \, [c_\lambda \ot x_\lambda] \ot x =
%\sum_{\lambda \in \Lambda_i} \, [c_\lambda \ot x_\lambda] \ot x_\lambda,
%\]