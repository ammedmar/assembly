% !TeX root = ../assembly.tex

\subsection{Introduction}\label{ss:introduction}

We begin with a quote from Andrew's ``Algebraic $L$-theory and topological manifolds.''
\begin{displaycquote}{ranicki1992topological}
	Generically, \textit{assembly} is the passage from a local input to a global output.
	The input is usually topologically invariant and the output is homotopy invariant.
	This is the case in the original geometric assembly map of Quinn, and the algebraic $L$-theory assembly map defined here.
\end{displaycquote}
In this note, we will consider the chain complex assembly of Andrew and M. Weiss \cite{ranicki1990assembly}, which can be extended to the $L$-theory assembly functors mentioned in the quote by considering chain complexes with derived Poincar\'e duality.
The connections to assembly via stable homotopy theory \cite{weiss1995asssembly} or equivariant homotopy theory \cite{davis1998assembly} will not be discussed.
In the context of this work, locality is encoded via a simplicial complex $X$, which for simplicity will now be assumed simply-connected.
We will focus on contravariant functors from $X$, regarded as a poset category, to the category of chain complexes $\Ch$.
The assembly of one such functor $\cN$ is its homotopy colimit, defined more precisely as its tensor product $\chains \Ot_X \cN$ with the covariant functor $\chains \colon X \to \Ch$ assigning to a simplex the chains on its closure subcomplex.
Andrew and M. Weiss factored the assembly functor
\[
\assembly \colon \Fun(X^\op, \Ch) \to \coMod_{\chains(X)} \to \Ch
\]
through the category of comodules over the Alexander--Whitney coalgebra of chains on $X$, and constructed, for any finite such comodule $N$, a zig-zag of comodule quasi-isomorphisms
\[
N \to N' \leftarrow \assembly \cN.
\]
For any two functors $\cN,\cN' \colon X^\op \to \Ch$ the chain map
\[
\Fun(X^\op,\Ch)(\cN,\cN') \to \Ch(\assembly \cN, \assembly \cN')
\]
induced by the assembly functor is not surjective in general.
That is to say, the assembly functor is not full.
Indeed, the morphisms in the domain are defined locally, whereas in the target they are global in nature.
The purpose of this article is to describe a subcategory $\coMod^{\Sym_2}_{\chains(X)}$ of $\coMod_{\chains(X)}$ and a further factorization
\[
\assembly \colon \Fun(X^\op, \Ch) \to
\coMod^{\Sym_2}_{\chains(X)} \to
\coMod_{\chains(X)} \to
\Ch
\]
with the first arrow a fully faithful functor.

Let us now motivate and describe the category $\coMod^{\Sym_2}_{\chains(X)}$.
For Andrew's $L$-theory assembly functors, it is important to provide the objects in $\Fun(X^\op, \Ch)$ with derived Poincar\'e duality.
For example, let us consider a (geometric) Poincar\'e duality complex $X$ with a preferred fundamental cycle, also denoted by $X$.
The Alexander--Whitney diagonal $\copr_0 \colon \chains(X) \to \chains(X) \ot \chains(X)$ defines a Poincar\'e duality quasi-isomorphism
\[
\begin{tikzcd}[column sep=tiny,row sep=0]
	\cochains^{-k}(X) \arrow[r] & \chains_{n-k}(X) \\
	\alpha \arrow[r,mapsto] & (\alpha \ot \id)\copr_0(X).
\end{tikzcd}
\]
Using $T\copr_0$ instead of $\copr_0$ above, where $T$ is the transposition of tensor factors, one obtains a new Poincar\'e duality quasi-isomorphism.
These two quasi-isomorphisms are different in general since $\copr_0$ and $T\copr_0$ are not equal, but they are chain homotopic.
Steenrod \cite{steenrod1947products} fully derived the $\Sym_2$ symmetry of the diagonal explicitly constructing natural ``higher diagonals'' $\copr_i$ with $\partial \copr_{i+1} = (1 \pm T) \copr_i$ and $\copr_{-1} = 0$.
The evaluation of the fundamental cycle on this richer structure provides $\chains(X)$ with the structure of a (symmetric) algebraic Poincar\'e complex as defined by Andrew.

We will focus on the Steenrod cup-$i$ coalgebra structure on $\chains(X)$ for $X$ not necessarily a Poincar\'e duality complex.
This structure can be expressed as an $\Sym_2$-equivariant chain map
\[
\copr \colon W \ot \chains(X) \to \chains(X) \ot \chains(X)
\]
where $W$ is the minimal free resolution of $\Z$ by $\Z[\Sym_2]$-modules.
We will show that the assembly of any functor $\cN \colon X^\op \to \Ch$ is equipped with a chain map
\[
\nabla \colon W \ot \assembly \cN \to \chains(X) \ot \assembly \cN
\]
specializing to the comodule structure of Andrew and M. Weiss.
The pair $(\assembly \cN, \nabla)$ is an object in the category $\coMod^{\Sym_2}_{\chains(X)}$ whose morphisms are given by linear maps $f \colon N \to N'$ making the diagram
\[
\begin{tikzcd}
	W \ot N \arrow[d, "\triangledown"] \arrow[r, "\id\, \ot f"] &
	W \ot N' \arrow[d, "\triangledown'"] & \\
	\chains(X) \ot N \arrow[r, "f \ot\, \id"] &
	\chains(X) \ot N'
\end{tikzcd}
\]
commute.
We can now state the contribution of this paper.

\begin{theorem*}
	Let $X$ be a simply-connected simplicial complex.
	The Ranicki--Weiss assembly functor to comodules over the Alexander--Whitney coalgebra of $X$ factors through comodules over its Steenrod cup-$i$ coalgebra
	\[
	\Fun(X^\op, \Ch) \to \coMod^{\Sym_2}_{\chains(X)} \to \coMod_{\chains(X)}
	\]
	with the first arrow a fully faithful functor and the second a forgetful one.
\end{theorem*}

This statement is 1-categorical, the $(\infty,1)$-categorical version will be discussed elsewhere.
In the $(\infty,1)$-categorical setting we mention the paper \cite{rivera2020system} where a connection between assembly and Brown's twisted tensor product is explored.
The goal of the theorem above is to provide an algebraic model of $\Fun(X^\op,\Ch)$ that retains its locality properties, serving as a possible step in the path connecting algebraic models of homotopy types \cite{quillen1969rational, sullivan1977infinitesimal,mandell2001padic} and Andrew's total surgery obstruction for topological manifold structures \cite{ranicki1979obstruction,ranicki1992topological,macko2013obstruction}.