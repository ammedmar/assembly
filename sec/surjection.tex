\documentclass[thesis.tex]{subfiles}

\subsection{The operad $\S$}

\begin{document}

This section collects results related to an $E_\infty$-operad studied by several researchers, whose combinatorial nature and explicit coaction on normalized chains makes it suitable for the applications of this work.

The coaction goes back to Steenrod construction in \cite{Ste47} of a chain approximation to the diagonal inclusion of triangulated spaces. The definition of an operad for which this coaction give rise to a natural coalgebra structure on the normalized chains of simplicial sets, appears in the proof of Deligne's conjecture by McClure-Smith \cite{MS02} and is treated under the name ``sequence operad" by the same authors in \cite{MS03}, where they present a filtration of it by $E_n$-suboperads. Work by Berger-Fresse in \cite{BF04} uses the same operad with the name ``surjection operad". Jonathan Potts' thesis \cite{Pot06} describes this operad with the name ``step operad" and Jones-Adamaszek relate it to join operations of augmented symmetric simplicial sets in \cite{AJ11}.

The definition of this $E_\infty$-operad, which will be denoted $\S$ to stand for Steenrod, sequence, surjection or step, will be presented below together with its filtration by $E_n$-suboperads and its natural coaction on the normalized chain complex of simplicial sets.

\paragraph{Chain complex of $\S$.} Let $\S(r)_d$ be the free abelian group generated by all functions from $\{1,\dotsc,r+d\}$ to $\{1,\dotsc,r\}$ quotiented by the submodule generated by all non-surjective functions and the surjections $u$ for which there exists $i\in\{1,\dotsc,r+d-1\}$ so that $u(i)=u(i+1)$. Set $\S(0)_0=\Z$ and $\S(0)_d=0$ for $d>0$. The module $\S(r)_d$ is free and any of the generators $u:\{1,\dotsc,r+d\}\to\{1,\dotsc,r\}$ will be identified with its ordered image $(u(1),\dotsc,u(r+d))$, which will be referred to as the \textbf{coordinates} of $u$.

To study examples a diagrammatical representation of the surjections proves useful, as an illustration that generalizes one has that $(1,2,1,3,2)$ is represented by
$$\xymatrix@C=1.5pc@R=1.5pc{
\t \ar@{-}[r]& \t\ar@{-}[d] &\t \ar@{-}[r]&\t \ar@{-}[d]& \t &  \\
\t &\t \ar@{-}[r]&\t \ar@{-}[u]&\t \ar@{-}[d]&\t \ar@{-}[r]&  \\
\t &\t &\t &\t \ar@{-}[r]& \t\ar@{-}[u]& . }$$

Define $\partial:\S(r)_d\to\S(r)_{d-1}$ by $$u\mapsto \sum_{i=1}^{r+d}\varepsilon^u_{i}\cdot(u(1),\dotsc,\widehat{u(i)},\dotsc,u(r+d))$$ with $\varepsilon^u_{i}$ a sign to be specified. In order to determine $\varepsilon^u_{i}$, separate the coordinates of $u$ into disjoint sets each characterized by one of the following properties:
\begin{enumerate}[a)]
\item The value of the coordinate equals the value of a coordinate to its right i.e. $u(i)=u(i+j)$ for some positive $j$.
\item The value of the coordinate is different from all coordinates to its right but equal to one on its left i.e. $u(i)\neq u(i+j)$ and $u(i)=u(i-k)$ for all $j$ and some $k$ positives.
\item The value of the coordinate is different from all other coordinates i.e. $u(i)\neq u(j)$ for any $i$.
\end{enumerate}
Consider the set $\{u(i_1), u(i_2),\dotsc\}$ of coordinates satisfying a) indexed so that $j<j'$ implies $i_j < i_{j'}$. Define for them $\varepsilon^u_{i_j}=(-1)^{j-1}$. For coordinates $u(i)$ satisfying b) define $\varepsilon^u_{i}=-\varepsilon^u_{i_j}$ with $u(i_j)=u(i)$ satisfying a) and $u(i_{j'})\neq u(i)$ for all $j'>j$. Coordinates $u(i)$ satisfying c) need no definition of $\varepsilon^u_{i}$ since $(u(1),\dotsc,\widehat{u(i)},\dotsc,u(r+d))=0.$

For example if $u=(1,2,1,3,2)$ then $u(1)$ and $u(2)$ satisfy a), $u(3)$ and $u(5)$ satisfy b) and $u(4)$ satisfy c) so
$$
\xymatrix@C=1pc@R=.5pc{
 \ \ \ar@{-}[r] & \t\ar@{-}[d] & \t\ar@{-}[r] & \t\ar@{-}[d] & \t & \t  \\
\partial \  & \t\ar@{-}[r] & \t\ar@{-}[u] & \t\ar@{-}[d] & \t\ar@{-}[r] &  \\
\t & \t & \t & \t\ar@{-}[r] & \t\ar@{-}[u] &  }
\xymatrix@C=1pc@R=.65pc{
 & \t & \t\ar@{-}[r] & \t\ar@{-}[d] & \t & \t  \\
= & \t\ar@{-}[r] & \t\ar@{-}[u] & \t\ar@{-}[d] & \t\ar@{-}[r] &  \\
\t & \t & \t & \t\ar@{-}[r] & \t\ar@{-}[u] &  }
\xymatrix@C=1pc@R=.5pc{
 \ \ \ar@{-}[r] & \t\ar@{-}[d] & \t & \t & \t  \\
- \  & \t\ar@{-}[r] & \t\ar@{-}[d] & \t\ar@{-}[r] &  \\
\t & \t & \t\ar@{-}[r] & \t\ar@{-}[u] &  }
\xymatrix@C=1pc@R=.5pc{
 \ \ \ar@{-}[r] & \t\ar@{-}[d] & \t\ar@{-}[r] & \t\ar@{-}[d] & \t \\
+ \  & \t\ar@{-}[r] & \t\ar@{-}[u] & \t\ar@{-}[d] &  \\
\t & \t & \t & \t\ar@{-}[r] & . }
$$

\paragraph{Operadic composition of $\S$.} The maps $$\circ_k:\S(r)_d\tensor\S(s)_e\to\S(r+s-1)_{d+e}$$ are defined as follows. Let $u\tensor v\in\S(r)_d\tensor \S(s)_e$ and let $\{i_1,\dotsc,i_n\}=u^{-1}(k)$. For any splitting $\pi$ of the coordinates of $v$ into $n$ subsequences
\begin{equation}
\big(v(l_0),\dotsc,v(l_1)\big) \ \big(v(l_1),\dotsc,v(l_2)\big) \ \cdot\cdot\cdot \ \big(v(l_{n-1}),\dotsc,v(l_n)\big)
\end{equation}
set the coordinates of a new surjection $u\circ_k^{\pi}v$ to be obtained from those of $u$ by first replacing $u(i_j)$ by the $j$-th subsequence of $v$, adding $k-1$ to those coordinates and then adding $s-1$ to the coordinates $u(i)$ which are greater than $n$. Define $$u\circ_k v=\sum_{\pi}\varepsilon_k^\pi\cdot(u\circ_k^{\pi}v)$$ with $\varepsilon_k^\pi$ a sign to be specified. In order to do so, notice that $u^{-1}(k)=\{i_1,\dotsc,i_n\}$ induces a collection of subsequences
\begin{equation}
\cdots\big(u(i_{j-1}+1),\dotsc,u(i_{j}-1)\big)\big(u(i_j)\big)\big(u(i_j+1),\dotsc,u(i_{j+1}-1)\big)\big(u(i_{j+1})\big)\cdots
\end{equation} and that in order to do the replacements in the definition of $u\circ_k^{\pi}v$ we can think of passing the subsequences from (1.1) across those from (1.2). The sign $\varepsilon_k^\pi$ will then be computed following the Koszul sign rule after defining the notion of degree for subsequences of coordinates.

Let $w$ be an arbitrary surjection and $$\cdots\big(w(m_{t-1}+1),\dotsc,w(m_t)\big)\big(w(m_t-1),\dotsc,w(m_{t+1})\big)\cdots$$ be the partition of $w$ into (consecutive and disjoint) subsequences determined by the coordinates $w(m_t)$ satisfying a). The degree of a general subsequence is then defined to be one less than the number of these subsequences that it overlaps with.

For example, to compute $u\circ_2 v=(2,1,3,2,1)\circ_2(1,2,1)$ we first compute the degrees of the relevant subsequences, which requires the counting of overlaps with $(2)(1)(3,2,1)$ for subsequences of $u$ and with $(1)(2,1)$ for those of $v$. Placing the degree as a subindex one obtains
$$\xymatrix@C=1pc@R=.5pc{ & (1)_0(1,2,1)_1 \\ (2)_0(1,3)_1(2)_0(1)_0 \ ; & (1,2)_1(2,1)_0 \\ & (1,2,1)_1(1)_0.} $$
Therefore,
$$\xymatrix@C=1pc@R=.5pc{
 & \t\ar@{-}[d]\ar@{-}[r] & \t & \t & \t\ar@{-}[r] & \t  \\
\ar@{-}[r] & \t & \t\ar@{-}[u]\ar@{-}[d] & \t\ar@{-}[d]\ar@{-}[r] & \t\ar@{-}[u] &  \\
\t & \t & \t\ar@{-}[r] & \t\ar@{-}[u] & \t &  }
\xymatrix@C=1pc@R=.5pc{
\ \ \ \ar@{-}[r]  & \t\ar@{-}[d] & \t\ar@{-}[d]\ar@{-}[r] &  \\
\circ_2\  & \t\ar@{-}[r] & \t &  \\
\ \ & \t & \t &}$$
equals$$\xymatrix@C=1pc@R=.61pc{
 &\t \ar@{-}[r]&\t \ar@{-}[d]&\t &\t &\t &\t \ar@{-}[r]&\t \\
\ \ \ar@{-}[r]&\t \ar@{-}[u]&\t \ar@{-}[d]&\t \ar@{-}[r]&\t \ar@{-}[d]&\t \ar@{-}[r]&\t \ar@{-}[u]&\t \\
+ \ &\t &\t \ar@{-}[d]&\t \ar@{-}[u]&\t \ar@{-}[r]&\t \ar@{-}[u]&\t &\t \\
 &\t &\t \ar@{-}[r]&\t \ar@{-}[u]&\t &\t &\t &\t}
\xymatrix@C=1pc@R=.61pc{
 &\t &\t \ar@{-}[r]&\t \ar@{-}[d]&\t &\t &\t \ar@{-}[r]&\t \\
\ \ \ar@{-}[r]&\t \ar@{-}[d]&\t \ar@{-}[u]&\t \ar@{-}[d]&\t &\t \ar@{-}[r]&\t \ar@{-}[u]&\t \\
- \  &\t \ar@{-}[r]&\t \ar@{-}[u]&\t \ar@{-}[d]&\t \ar@{-}[r]&\t \ar@{-}[u]&\t &\t \\
 &\t &\t &\t \ar@{-}[r]&\t \ar@{-}[u]&\t &\t &\t}
\xymatrix@C=1pc@R=.61pc{
 &\t &\t &\t \ar@{-}[r]&\t \ar@{-}[d]&\t &\t \ar@{-}[r]&\t \\
\ \ \ar@{-}[r]&\t \ar@{-}[d]&\t \ar@{-}[r]&\t \ar@{-}[u]&\t \ar@{-}[d]&\t \ar@{-}[r]&\t \ar@{-}[u]&\t \\
- \ &\t \ar@{-}[r]&\t \ar@{-}[u]&\t &\t \ar@{-}[d]&\t \ar@{-}[u]&\t &\t \\
 &\t &\t &\t &\t \ar@{-}[r]&\t \ar@{-}[u]&\t & .}$$

\paragraph{Symmetric action on $\S$.} Define an action of $\Sigma_r$ on $\S(r)_d$ by $$\sigma\cdot u=\big(\sigma(u(1)),\dotsc,\sigma(u(r+d))\big).$$\par
For example,
$$\xymatrix@C=1pc@R=.45pc{
\ \ \ \ \ \ar@{-}[r] & \t\ar@{-}[d] & \t\ar@{-}[r] & \t\ar@{-}[d] & \t & \t  \\
(123) & \t\ar@{-}[r] & \t\ar@{-}[u] & \t\ar@{-}[d] & \t\ar@{-}[r] &   \\
 & \t & \t & \t\ar@{-}[r] & \t\ar@{-}[u] & }
\xymatrix@C=1pc@R=.8pc{
\ \ &\t &\t &\t \ar@{-}[r]&\t \ar@{-}[d]&\t  \\
= \ \ar@{-}[r]&\t \ar@{-}[d]&\t \ar@{-}[r]&\t \ar@{-}[u]&\t \ar@{-}[d]&\t  \\
 &\t \ar@{-}[r]&\t \ar@{-}[u]&\t &\t \ar@{-}[r]&\t
}$$

\begin{lemma}
The structure defined above makes $\S=\{\S(r)_\bullet\}$ into an $E_{\infty}$-operad (see Definition \ref{E infinity operad}).
\begin{proof}
The action of $\Sigma_r$ on $\S(r)_\bullet$ is free since it is free on the first coordinate of any surjection. The proof that $\S(r)_\bullet$ has the homology of a point is part $(c)$ of Theorem 2.15 in \cite{MS03}.
\end{proof}
\end{lemma}

\paragraph{Filtration of $\S$ by $E_n$-operads}

Consider a surjection $u$ in $\S(r)_\bullet$ and a pair $i<j\leq r$. Define $u_{ij}$ to be the sequence obtained from the sequence of coordinates of $u$ by removing all elements different from $i$ and $j$. For example, if $u=(2,1,3,1,2)$ then $u_{12}=(2,1,1,2)$, $u_{13}=(1,3,1)$ and $u_{23}=(2,3,2)$. To each such sequence assign the number of pairs of distinct consecutive coordinates and name it the \textbf{change number}. Using the previous example one sees that the change number of all $u_{ij}$ is $2$. For any $u$ define its \textbf{filtration weight} to be the largest change number among all possible $u_{ij}$ and define $\S^n$ to be the suboperad generated by all surjections whose filtration weight is less than or equal to $n$.

The following appears as Theorem 3.5 in \cite{MS03}.
\begin{lemma}\label{filtration by E_n}
The constructions above defines a filtration by suboperads
$$\S^1\leq\S^2\leq\dotsb\leq\S^\infty=\S$$
with $\S^n$ an $E_n$-operad.
\end{lemma}

\paragraph{$\S$-coalgebra structure on normalized chains.} Let $\Delta$ be the simplicial category as described in Definition \ref{Simplicial category and simplicial sets} and recall from Example \ref{normalized chain complex} the functor $\chains:\Delta\to\Ch$ whose left Kan extension along the Yoneda embedding defines the normalized chains of simplicial sets. The purpose of this section is to construct a compatible collection of maps
$$\S(k)\tensor\chains[0,\dotsc,n]\to\chains[0,\dotsc,n]^{\tensor k},$$
indexed by $k,n\geq0$, determining a functor represented as a dotted arrow in the following commutative diagram
$$\xymatrix{ & \coAlg_{\S}\ar[d]^{\text{forget}}\\ \sSet \ar[r]_{\chains}\ar@{-->}[ru] & \Ch.}$$

Let $[i_1,\dotsc,i_l]$ be the image of an order preserving function $[0,\dotsc,l]\to[0,\dotsc,n]$. This function induces a chain map $\chains [0,\dotsc,l]\to\chains [0,\dotsc,n]$ and the image the top dimensional generator of $\chains [0,\dotsc,l]$ will be identified with the generator $[i_1,\dotsc,i_l]$ in $\chains [0,\dotsc,n]$ if $i_j\neq i_{j+1}$ for all $j$, being $0$ otherwise.

Let $u\in\S(r)_d$ be a surjection and $[0,\dotsc,n]\in\Delta$. Let $\pi$ stand for a choice of $(r+d-1)$ elements of $\{0,\dotsc,n\}$ satisfying $$0=n_0\leq n_1\leq\dotsb\leq n_{r+d-1}\leq n_{r+d}=n$$
and associate to this choice $\pi$ a collection of generators of $\chains [0,\dotsc,n]$
$$\big\{[n_{i-1},\dotsc,n_i]\big\}_{i=0}^{r+d}$$
with consecutive vertices. Such generators will be referred to as \textbf{intervals}.

For $k\leq r$ define $L_{\pi}(k)$ to be $0$ in case there exists a pair of intervals $[n_{i-1},\dotsc,n_i]$ and $[n_{j-1},\dotsc,n_j]$ with $u(i)=u(j)=k$ and a common vertex, or define $L_{\pi}(k)$ to be the generator in $\chains[0,\dotsc,n]$ whose set of vertices is the union of the vertices of all intervals $[n_{i-1},\dotsc,n_i]$ with $u(i)=k$.

For every choice of $\pi$ define an element in $\chains[0,\dotsc,n]^{\tensor r}$ by $$u_\pi[0,\dotsc,n]=L_{\pi}(1)\tensor L_{\pi}(2)\tensor\dotsb\tensor L_{\pi}(r)$$
and set
$$u[0,\dotsc,n]=\sum_\pi\varepsilon_{u,\pi}\cdot u_\pi[0,\dotsc,n]$$ with $\varepsilon_{u,\pi}$ a sign to be specified.

\begin{example} \label{Delta_0 (no signs)}
If $u=(1,2)$, the value of $u[0,1,..,n]$ is equal to
\begin{align*}
\xymatrix@C=1.2pc@R=1.2pc{
\ar@{-}[r] &\t\ar@{-}[d] & \\
\t &\t \ar@{-}[r]& } \begin{matrix}\\ [0,1,\dotsc,n]\ \\ \end{matrix} & \begin{matrix}\\ =\ \sum_i\,[0,\dotsc,i]\tensor [i,\dotsc,n] \\ \end{matrix} \\
&  \begin{matrix}\\ =\  \\ \end{matrix}\xymatrix@C=.01pc@R=.1pc{0& \\ 0&1 \ \dotsc \ n } \begin{matrix}\\ \ + \ \\ \end{matrix} \xymatrix@C=.01pc@R=.1pc{0& 1 &\\\ & 1 &\dotsc \ n } \begin{matrix}\\ \ + \cdots \\ \end{matrix} \begin{matrix}\\ \ + \ \\ \end{matrix} \xymatrix@C=.01pc@R=.1pc{0\ 1 \ \dotsc & n \\ & \,n,}
\end{align*}
with signs computed to be all positive in Example \ref{Delta_0 (the signs)}.
\end{example}

\paragraph{Signs of the $\S$-coalgebra structure}
In order to specify the sign $\varepsilon_{u,\pi}$ one distinguishes between two types of intervals.
\begin{enumerate}[a)]
\item \textbf{Internal} intervals $[n_{i-1},\dotsc,n_i]$ satisfy $u(i)=u(i+j)$ for some positive~$j$.
\item \textbf{Final} intervals $[n_{i-1},\dotsc,n_i]$ satisfy $u(i)\neq u(i+j)$ for all positive $j$.
\end{enumerate}
Define the \textbf{degree} of an interval $[n_{i-1},\dotsc,n_i]$ to be $n_i-n_{i-1}+1$ if it is internal or $n_i-n_{i-1}$ if it is final.

Consider the permutation taking $$(u(1),u(2),\dotsc,u(r+d))\mapsto(1,\dotsc,1,2,\dotsc,2,\dotsc,r,\dotsc,r)$$ and obtain, by Koszul's rule, a sign $\varepsilon_{u,\pi}^{per}$ from the induced permutation of the graded intervals
$$\big([0,\dotsc,n_1],[n_1,\dotsc,n_2],\dotsc,[n_{r+d-1},\dotsc,n_{r+d}]\big).$$

Consider all internal intervals $[n_{i-1},\dotsc,n_i]$ to be the indexing set of the sum $\sum n_i$. Let this sum be the exponent of a sign $\varepsilon_{u,\pi}^{pos}$ and set
$$\varepsilon_{u,\pi}=\varepsilon_{u,\pi}^{per}\cdot\varepsilon_{u,\pi}^{pos}.$$

\begin{example}\label{Delta_0 (the signs)}
Let $u=(1,2)$ as in Example \ref{Delta_0 (no signs)}. For any $[0,\dotsc,n]$ and any $\pi$, all signs $\varepsilon_{u,\pi}$ are positive since every $[n_{i-1},\dotsc,n_i]$ is final, so $\varepsilon_{u,\pi}^{pos}=1$, and $\varepsilon_{u,\pi}^{per}=1$ because $(1,2)$ is already in the correct order.
\end{example}

\begin{example} \label{Delta_n on n-simplices (with signs)}
Let $u=(\dotsc,2,1,2)$ be one of the two generators of $\S(2)_d$ and consider $[0,\dotsc,d]\in\Delta$ of the same dimension as the degree of $u$.

In order to compute the coaction of $u$ on $[0,\dotsc,d]$, thought of as one of the generators of $\chains[0,\dotsc,d]$, one notices that the only choice for $\pi$ $$0=n_0\leq n_1\leq\dotsb\leq n_{r+d-1}\leq n_{r+d}=d$$ leading to a non-zero $u_{\pi}[0,\dotsc,d]$ satisfies $n_i\neq n_{i+1}$ for all  $i=1,\dotsc,d$. Because of the relation between the degree of the surjection and the dimension of the simplex, this choice is unique and given by $n_i=i-1$ for all $i=1,\dotsc,d$, in other words
$$0\leq0<1<2<\dotsb<(d-1)<d\leq d,$$
so up to a sign $\varepsilon_d$ one has \vspace*{-30pt}
\begin{align*}
\begin{matrix}\ \\ \\ \dotsb \end{matrix}
\xymatrix@C=1pc@R=1.8pc{
\ar@{-}[r] &\t\ar@{-}[d] &\t \ar@{-}[r]&\t \ar@{-}[d]&\t \\
\t &\t \ar@{-}[r]&\t \ar@{-}[u]&\t \ar@{-}[r]&\t }\ \begin{matrix}\ \\ \ \\ \Big([0,1,\dotsc,d]\Big) \ \ =\ \varepsilon_d\ \cdot\ \end{matrix}&\
\begin{matrix}\\ \\ 0 &1 &\dotsc &d \\ 0 &1 &\dotsc &d\end{matrix} \\
\begin{matrix}\ \\ =\ \varepsilon_d\ \cdot\ \end{matrix}&\
\begin{matrix}\ \\  [0,1,\dotsc,d]\tensor[0,1,\dotsc,d]. \end{matrix}
\end{align*}

In order to determine the sign $\varepsilon_d$ notice that only the last two intervals $[d-1,d]$ and $[d,d]$ are final with degrees $1$ and $0$ respectively. All other intervals $[i-1,i]$ are internal and have degree $2$ except for $[0,0]$ which has degree $1$. Therefore, a permutation contributes with a negative sign if and only if it exchanges $[0,0]$ and $[d-1,d]$.
Consequently, the permutation $$(\dotsc,2,1,2)\mapsto(1,1,\dotsc,2,2)$$ induces the permutation sign $\varepsilon_d^{per}=(-1)^d$. The position sign $\varepsilon_{d}^{pos}$, determined by the internal intervals, is equal in this case to $\sum_{i=0}^{d-1}i$ so $$\varepsilon_d=\varepsilon_d^{per}\cdot\varepsilon_{d}^{pos}=(-1)^d\cdot\sum_{i=0}^{d-1}i=(-1)^{d(d+1)/2}.$$
\end{example}

\begin{example}
Let $u=(1,2,3,1)$ and $[0,1,2]\in\Delta$, this example will compute $u[0,1,2]$. A choice of
$$\pi:0=n_0\leq n_1\leq n_2\leq n_3\leq n_4=2$$
leads to a non-zero term $u_{\pi}$ if and only if $n_1\neq n_3$. The following table summarizes for such possible choices the degrees of the associated internal and final intervals, as well as the resulting permutation and position signs.

\begin{center}
\setlength{\tabcolsep}{4.5pt}
  \begin{tabular}{ | c | c | c || r | r | r | r || c || c ||}
  \hline
  $n_1$ &$n_2$ &$n_3$ &$|[n_0,\dotsc,n_1]|$ &$|[n_1,\dotsc,n_2]|$ &$|[n_2,\dotsc,n_3]|$ &$|[n_3,\dotsc,n_4]|$ &$\varepsilon^{per}$ &$\varepsilon^{pos}$ \\ \hline
  0     &0      &1      &1 \ \ (i)             &0 \ \ (f)          &1 \ \ (f)          &1 \ \ (f)          & -1 &\ 1 \\ \hline
  0     &0      &2      &1 \ \ (i)             &0 \ \ (f)          &2 \ \ (f)          &0 \ \ (f)          &\ 1 &\ 1 \\ \hline
  0     &1      &1      &1 \ \ (i)             &1 \ \ (f)          &0 \ \ (f)          &1 \ \ (f)          & -1 &\ 1 \\ \hline
  0     &1      &2      &1 \ \ (i)             &1 \ \ (f)          &1 \ \ (f)          &0 \ \ (f)          &\ 1 &\ 1 \\ \hline
  0     &2      &2      &1 \ \ (i)             &2 \ \ (f)          &0 \ \ (f)          &0 \ \ (f)          &\ 1 &\ 1 \\ \hline
  1     &1      &2      &2 \ \ (i)             &0 \ \ (f)          &1 \ \ (f)          &0 \ \ (f)          &\ 1 & -1 \\ \hline
  1     &2      &2      &2 \ \ (i)             &1 \ \ (f)          &0 \ \ (f)          &0 \ \ (f)          &\ 1 & -1 \\ \hline
  \end{tabular}
\end{center}
Therefore,
$$\xymatrix@C=1.5pc@R=1pc{
 \t \ar@{-}[r]&\t \ar@{-}[d]&\t &\t \ar@{-}[r]& \ \ \ \ \ \ \ \\
 &\t \ar@{-}[r]&\t \ar@{-}[d]&\t \ar@{-}[u]& \ [0,1,2]\\
 &\t &\t \ar@{-}[r]&\t \ar@{-}[u]& \\}$$
equals
$$\begin{matrix} &0 &1 &2 \\ -& 0& & \\ & 0& 1&\end{matrix}\ \ \ \begin{matrix} &0 & &2 \\ +& 0& & \\ & 0& 1& 2\end{matrix}\ \ \ \begin{matrix} &0 &1 &2 \\ -& 0& 1& \\ & & 1&\end{matrix}$$\ \par$$\begin{matrix} &0 & &2 \\ +& 0& 1& \\ & & 1& 2\end{matrix}\ \ \ \begin{matrix} &0 & &2 \\ +& 0& 1& 2\\ & & & 2\end{matrix}\ \ \ \begin{matrix} &0 &1 &2 \\ -& & 1& \\ & & 1& 2\end{matrix}\ \ \ \begin{matrix} &0 &1 &2 \\ -& & 1& 2\\ & 0& &\ 2.\end{matrix}$$
\end{example}

\begin{lemma} \label{chains as S-coalgebra}
The maps define above
$$\S(k)\tensor\chains[0,\dotsc,n]\to\chains[0,\dotsc,n]^{\tensor k}$$
determine a functor which is represented by the dotted arrow in the following commutative diagram $$\xymatrix{\Delta \ar@{-->}[r]\ar[rd]_{\chains} & \coAlg_{\S}\ar[d]^{\text{forget}}\\ & \Ch.}$$
\begin{proof}
This follows from part $(b)$ of Theorem 2.15 in \cite{MS03}, see also part $(b)$ of Remark 2.16 in the same reference.
\end{proof}
\end{lemma}

\begin{definition}\label{coalgebra in normalized chains}
As in Definition \ref{realization and nerve}, a functor can be constructed by taking the left Kan extension along the Yoneda embedding of the functor of Lemma \ref{chains as S-coalgebra}. Such functor is represented by the dotted arrow in the following commutative diagram
$$\xymatrix{ & \coAlg_{\S}\ar[d]^{\text{forget}}\\ \sSet \ar[r]_{\chains}\ar@{-->}[ru] & \Ch}$$
and the image of any simplicial set $X$ by this functor will be referred to as the \textbf{$\S$-coalgebra structure on} $\chains(X)$.

For any $1\leq k\leq\infty$, let $\S^k(2)$ denote the suboperad generated by the arity $2$ part of the $k$-level of the filtration described in Lemma \ref{filtration by E_n}. Composing the above functor with the forgetful functor one has
$$\xymatrix{ & \coAlg_{\S}\ar[d]^{\text{forget}}\\ \sSet \ar@{-->}[r]_-{\chains}\ar[ru]^-{\chains} & \coAlg_{\S^k(2),}}$$
and the image of any simplicial set $X$ by this functor will be referred to as the \textbf{$\S^k(2)$-coalgebra structure on} $\chains(X)$.
\end{definition}

\begin{notation}\label{notation for Deltas}
Let $X$ be a simplicial set and consider the $\S$-coalgebra structure on $\chains(X)$. For every surjection $u\in S(k)_\bullet$ one has an abelian group homomorphism $$\chains(X)\to\chains(X)^{\tensor k}.$$
For $u=(\dotsc,2,1,2)$ one of the generators of $\S(2)_d$, the associated map will be denoted
$$\Delta_d:\chains(X)\to\chains(X)\tensor\chains(X).$$
 \end{notation}


\end{document}