\documentclass[thesis.tex]{subfiles}

\begin{document}

\section{Abelian sheaves and $\S$-comodules}

\subsection{Sheaf theory of posets}

This section is divided into three parts. The first part builds on a know equivalence of categories between partially ordered sets and $T_0$-Alexandrov spaces. Each of these categories is equipped with a duality, opposite poset and topology of closed sets respectively, and Lemma \ref{Alexandorv spaces, posets and their dualities} shows that the categorical equivalence is equivariant with respect to these dualities.

The second part uses the connection between posets and Alexandrov spaces to relate the categorical definition of sheaves and cosheaves with the topological one, presenting the close connection that arises between sheaves and cosheaves over posets.

The third part specializes to sheaves and cosheaves over posets with values in an abelian category and characterizes their projective objects, showing the existence of enough projectives under suitable conditions.

\paragraph{Alexandrov spaces and partially ordered sets}
\begin{definition}(Alexandrov Spaces) A topological space $(X,\tau)$ is said to be an \textbf{Alexandrov space} if an arbitrary intersection of open sets is an open set. This extra condition allows for the definition of a new topology on any Alexandrov space given by all closed sets of the original topology. This topology will be called the \textbf{dual topology} an denoted $\tau^\c$.

The full subcategory of topological spaces given by Alexandrov spaces satisfying the $T_0$ separation axiom (i.e.\! for any pair of points there is an open set containing one of them, but not both) will be denoted $\AT_0$. Notice that $(X,\tau)\in\AT_0$ if and only if $(X,\tau^{\c})\in\AT_0$.
\end{definition}

\begin{definition}
Let $(P,\leq)$ be a poset. The set $P$ can be made into a topological space in two ways.
Define the topology $\tau_\geq$ in $P$ to be generated by the subsets $b_\geq\stackrel{\mathrm{def}}{=}\{a:b\geq a\}$ for all $b\in P$, and the topology $\tau_\leq$ to be generated by the subsets $b_\leq\stackrel{\mathrm{def}}{=}\{c:b\leq c\}$ for all $b\in P$.\vspace*{6pt}

Let $(X,\tau)$ be a $T_0$-Alexandrov space. The set $X$ can be made into a poset in two ways.
For any $x\in X$ let $U_x\stackrel{\mathrm{def}}{=}\bigcap_{x\in U\in\tau}U$.
Define $X_\subset=(X,\leq)$ with $x\leq y$ if and only if $U_x\subset U_y$, and define $X_\supset=(X,\leq)$ with $x\leq y$ if and only if $U_x\supset U_y$.
\end{definition}

\begin{lemma}\label{from posets to A spaces and back}
The four assignments described above are functorial. Moreover, they define pairs of inverse functors
$$(-)_{\geq}:\Poset\leftrightarrows\AT_0:(-)_\subset$$
and
$$(-)_{\leq}:\Poset\leftrightarrows\AT_0:(-)_\supset.$$
\begin{proof}
Given an order preserving function $f:P\to P'$ one needs to prove that $f$ is continuous with respect to both topologies. Let $c'_\geq$ be a basis element of $\tau'_\geq$ and consider $f^{-1}(c'_\geq)$. This set is open since it is straightforward to check that
$$f^{-1}(c'_\geq)\,\,=\!\!\!\bigcup_{\{b:\,c'\geq f(b)\!\}}\!\!b_\geq.$$
Analogously, for a basis element $a'_\leq$ of $\tau'_\leq$ one has
$$f^{-1}(a'_\leq)\,\,=\!\!\!\bigcup_{\{b:\,a'\leq f(b)\!\}}\!\!b_\leq.$$

Given a continuous function $f:X\to X'$ one needs to prove that if $U_x\subset U_y$ then $U_{f(x)}\subset\, U_{f(y)}$. To do so, notice that $f^{-1}(U_{f(y)})$ is an open set containing $y$ and, since $U_y$ is the smallest open set with that property, $U_y\subset f^{-1}(U_{f(y)})$. The assumption $U_x\subset U_y$ together with the previous observation imply that
$$f(x)\in f(U_x)\subset f(U_y)\subset U_{f(y)},$$
which, since $U_{f(x)}$ is the smallest open set containing $f(x)$, give the desired $U_{f(x)}\subset\, U_{f(y)}$.

Verifying these pairs of functors are inverse of each other follows directly from noticing that in $\tau_\geq$ one has $U_b=b_\geq$, while in $\tau_\leq$ one has $U_b=b_\leq$.
\end{proof}
\end{lemma}

\begin{definition}
The (covariant) functor associating to a $T_0$-Alexandrov space the $T_0$-Alexandrov space with the dual topology is denoted by
$$(-)^\c:\AT_0\to\AT_0.$$
The (covariant) functor associating to a poset the poset with the opposite order is denoted by
$$(-)^{\op}:\Poset\to\Poset.$$
\end{definition}

\begin{lemma} \label{Alexandorv spaces, posets and their dualities}
The functors defined in this section are related by the following identities:\vspace*{5pt}\par
\begin{minipage}[l]{7cm}
\begin{enumerate}
\item[1a.] $(-)_\geq=(-)_\leq\circ(-)^{\op}.$
\item[2a.] $(-)_\geq=(-)^{\c}\circ(-)_\leq.$
\item[3a.] $(-)_\subset=(-)^{\op}\circ(-)_\supset.$
\item[4a.] $(-)_\subset=(-)_\supset\circ(-)^{\c}.$
\end{enumerate}
\end{minipage}
\begin{minipage}{5cm}
\begin{enumerate}
\item[1b.] $(-)_\leq=(-)_\geq\circ(-)^{\op}.$
\item[2b.] $(-)_\leq=(-)^{\c}\circ(-)_\geq.$
\item[3b.] $(-)_\supset=(-)^{\op}\circ(-)_\subset.$
\item[4b.] $(-)_\supset=(-)_\subset\circ(-)^{\c}.$
\end{enumerate}

\end{minipage}

\begin{proof}
Pairs of corresponding identities are equivalent since $$(-)^{\op}\circ(-)^{\op}=\id\text{ and }(-)^{\c}\circ(-)^{\c}=\id.$$
The third pair of identities follows from the first pair since $$(-)_\geq\circ(-)_\subset=\id\text{ and }(-)_\supset\circ(-)_\leq=\id.$$
The fourth pair of identities follows from the second pair since $$(-)_\subset\circ(-)_\geq=\id\text{ and }(-)_\leq\circ(-)_\supset=\id.$$
\textit{Proof of 1a.}\! For any poset $(P,\leq)$ the topology $\tau_\geq$ is generated by sets $\{a:b\geq a\}$, while the topology $\tau_{\leq^{\op}}$ is generated by sets $\{a:b\leq^{\op} a\}$. These bases are equal so $\tau_\leq=\tau_{\geq^{\op}}$.\vspace*{3pt}\\
\textit{Proof of 2a.} \!Consider an arbitrary poset $(P,\leq)$. It will be shown first that $\tau_\geq\subset\tau_\leq^\c$. To do so, consider a generator $\{a:b\geq a\}$ of $\tau_\geq$ and notice that
\begin{equation*}
\{a:b\geq a\}\,\subset\left(\bigcup_{\{x\,:\,b\ngeq x\}}\{y:x\leq y\}\right)^\c\!\!\in\tau_\leq^\c
\end{equation*}
since $b\geq a$ and $a\geq x$ implies $b\geq x$. Also, one has
$$\{a:b\geq a\}^\c\ \subset\bigcup_{\{x\,:\,b\ngeq x\}}\{y:x\leq y\}\vspace*{-8pt}$$
since $x\in\{a:b\geq a\}^\c$ implies $x\in\{y:x\leq y\}$ with $b\ngeq x$.

In order to show that $\tau_\geq\supset\tau_{\leq}^\c$ consider an arbitrary open set of $\tau_{\leq}^\c$ say $\big(\bigcup_{x\in I}\{y:x\leq y\}\big)^\c=\bigcap_{x\in I}\{y:x\leq y\}^\c$. Since $\tau_\geq$ is closed under arbitrary intersections, it suffices to show that $\{y:x\leq y\}^\c$ is open in $\tau_\geq$. This follows from a computation similar to the one above showing that
\begin{equation*}
\{y:x\leq y\}^\c\,=\!\!\bigcup_{\{b\,:\,x\nleq b\}}\{a:b\geq a\}\in\tau_\geq, \vspace*{-8pt}
\end{equation*}
and concludes the proof.
\end{proof}
\end{lemma}

\paragraph{Sheaves, cosheaves and their relationship over posets}

\begin{definition}(Presheaves and precosheaves)
Let $\C$ and $\V$ be categories. A \textbf{presheaf}, respectively \textbf{precosheaf}, on $\C$ with values on $\V$ is a
contravariant, respectively covariant, functor from $\C$ to $\V$. A \textbf{presheaf morphism}, respectively \textbf{precosheaf morphism}, is a natural transformation of such functors.

Denote these categories respectively by $\PSh(\C,\V)$ and $\PcoSh(\C,\V)$, or if $\V$ is understood from the context, simply by $\PSh(\C)$ and $\PcoSh(\C)$.
\end{definition}

\begin{definition}(Sites)
A \textbf{site} is given by a (small) category $\C$ and a set $\Cov(\C)$ of families of
morphisms with fixed target $\{U_i\to U\}_{i\in I}$, called \textbf{coverings} of C, satisfying the
following axioms:
\begin{enumerate}[S1:]
\item If $V\to U$ is an isomorphism then $\{V\to U\}\in\Cov(\C)$.
\item If $\{U_i\to U\}_{i\in I}\in\Cov(\C)$ and for each $i\in I$ one has that $\{V_{ij}\to U_i\}_{j\in J_i}\in\Cov(\C)$,
    then $\{V_{ij}\to U\}_{i\in I, j\in J_i}\in\Cov(\C)$.
\item If $\{U_i\to U\}_{i\in I}\in\Cov(\C)$ and $V\to U$ is a morphism in $\C$ then the pullback $U_i\times_U V$ exists for each $i\in I$ and $\{U_i\times_U V\to V\}_{i\in I}\in\Cov(\C)$.
\end{enumerate}
\end{definition}

\begin{example}\label{open sets as a site}
Let $(X,\tau)$ be a topological space. Think of $\tau$ as a category with set of objects $\tau$ and $$\Hom_\tau(V,U)=\begin{cases}\{V\to U\}& \text{ if } V\subset U,\\ \hspace*{.81cm} \emptyset & \text{ if } V\not\subset U,\end{cases}$$
and notice that this assignment $\Top\to\Cat$ is functorial. Define the set of coverings of $\tau$ by
$$\{U_i\to U\}_{i\in I}\in\Cov(\tau) \text{ if and only if } \bigcup_{i\in I}U_i=U.\vspace*{-3pt}$$
The conditions for $\tau$ with this coverings to define a site are easily verified.
\end{example}

\begin{remark}\label{rmk: Poset -> Top -> site}
The functors $(-)_\geq$ and $(-)_\leq$ from Lemma \ref{from posets to A spaces and back} can be composed with the functor described in the previous example. Abusing notation, these resulting functors are denoted
$$\begin{array}{cc}
(-)_\geq:\!\! & \Poset\to\Cat \\
 & \!\!(P,\leq)\mapsto\,\tau_\geq.
\end{array} \text{\ \ \ \ \  \& \ \ \ } \begin{array}{cc}
(-)_\geq:\!\! & \Poset\to\Cat \\
 & \!\!(P,\leq)\mapsto\,\tau_\geq.
\end{array}$$
\end{remark}

\begin{example}\label{indiscrite site on a poset}
Given any (small) category, define a site by declaring the coverings to be the identity morphisms only. In particular, since the assignment that takes any poset $(P,\leq)$ to a category with set of objects $P$ and morphisms $$\Hom_P(x,y)=\begin{cases}\{x\to y\}& \text{ if } x\leq y\\ \ \ \ \emptyset & \text{ if } x\nleq y\end{cases}$$ is a full and faithful functor, any poset $P$ can be thought of a site with trivial coverings.
\end{example}

The following definitions use some of the examples of limits and colimits described in Appendix A.

\begin{definition}(Sheaves and cosheaves)
Let $\C$ be a site, $\D\in\PSh(\C)$ and $\E\in\PcoSh(\C)$. The presheaf $\D$ is said to be a \textbf{sheaf} if for all $\{U_i\to U\}_{i\in I}\in\Cov(\C)$ the first arrow in the following diagram represents the equalizer of the next two $$\D(U)\to\prod_{i\in I}\D(U_i)\rightrightarrows\prod_{i,j\in I}\D(U_i\times_U U_j).$$ The full subcategory of sheaves in $\PSh(\C)$ will be denoted by $\Sh(\C)$.\par

The precosheaf $\E$ is said to be a \textbf{cosheaf} if for all $\{U_i\to U\}_{i\in I}\in\Cov(\C)$ the last arrow in the following diagram represents the coequalizer of the first two $$\coprod_{i,j\in I}\E(U_i\times_U U_j)\rightrightarrows\coprod_{i\in I}\E(U_i)\to\E(U).$$
The full subcategory of cosheaves in $\PcoSh(\C)$ will be denoted by $\coSh(\C)$.
\end{definition}

\begin{example}
Let $(X,\tau)$ be a topological space. Sheaves and cosheaves on the site $\tau$, defined in Example \ref{open sets as a site}, agree with the usual topological sheaves and cosheaves on $X$.
\end{example}

\begin{example}
If a site has coverings given by the identity morphisms only, then the categories of sheaves and presheaves on such site agree; as also do the categories of cosheaves and precosheaves on it. In particular, following Example \ref{indiscrite site on a poset}, for any poset $(P,\leq)$ one has $\Sh(P)=\PSh(P)$ and $\coSh(P)=\PcoSh(P)$ over this indiscrete site.
\end{example}

\begin{definition}
Let $(P,\leq)$ be a poset and $\V$ a complete and cocomplete category. From Remark \ref{rmk: Poset -> Top -> site} one obtains a covariant functor $(-)_{\geq}:P\to\tau_\geq$ when $P$ is regarded as a category. Define the functor
$$\Lan:\Sh(P,\V)\to\PSh(\tau_{\leq},\V),$$
which assigns to any $\D\in\Sh(P,\V)$ the left Kan extension of $\D$ along $(-)^{\op}_{\geq}$, diagrammatically
$$\xymatrix{P^{\,\op}\ar[r]^{\D}\ar[d]_{(-)^{\op}_\geq} & \V \\
\ \ (\tau_\geq)^{\op}. \ar@{-->}[ur]_-{\Lan\D} & }$$
The functor $\Ran:\coSh(P^{\op},\V)\to\PcoSh(\tau_{\leq},\V)$ is defined using the contravariant functor $(-)_{\leq}:P\to\tau_\leq$ in a similar manner, diagrammatically
$$\xymatrix{P^{\,\op}\ar[r]^{\E}\ar[d]_{(-)_\leq} & \V \\ \ \ \tau_\leq. \ar@{-->}[ur]_-{\Ran\E} & }$$
\end{definition}

\begin{lemma}
Let $(P,\leq)$ be a poset and $\V$ a complete and cocomplete category. For any $\D\in\Sh(P,\V)$ the presheaf $\Lan\D$ is a sheaf and the functor $$\Lan:\Sh(P,\V)\to\Sh(\tau_{\geq},\V)$$ is an equivalence of categories.
Similarly, for any $\E\in\coSh(P^{\op},\V)$ the precosheaf $\Ran\E$ is a cosheaf and the functor $$\Ran:\coSh(P^{\op},\V)\to\coSh(\tau_{\leq},\V)$$ is an equivalence of categories.
\begin{proof}
One needs to verify that the presheaf $\Lan\D$ satisfies the sheaf condition. For any $U\in\tau_\geq$, Lemma \ref{formula for Kan extensions} and Lemma \ref{formula for lim} provide a formula for the left Kan extension
$$\Lan\D(U)=\eq\big(\!\!\!\prod_{\,\,\,y_\geq \subset\, U\ }\!\!\!\D(y)\rightrightarrows\!\!\!\!\prod_{\,x_\geq\subset\, y_\geq}\!\!\!\D(x)\big),$$
which is exactly the sheaf condition for the finest cover of $U$. (Recall that the collection $\{y_\geq:y\in P\}$ forms a basis of $\tau_\geq$). The inverse functor of $\Lan$ is given by restricting a sheaf on $\tau_\geq$ to the basis $\{y_\geq:y\in P\}$.

The proof for cosheaves is analogous using Lemma \ref{formula for Kan extensions} and Lemma \ref{formula for colim}.
\end{proof}
\end{lemma}

\begin{definition}
Let $(P,\leq)$ be a poset. Consider the contravariant functor $(-)^{\c}:\tau^{\c}_\geq\to(\tau_\geq)$ taking $U^{\c}$ to $U$ and recall from Lemma \ref{Alexandorv spaces, posets and their dualities} that $\tau_\geq^{\c}=\tau_\leq$. Abusing notation, define the functor $$(-)^\c:\Sh(\tau_\geq)\to\PcoSh(\tau_\leq)$$
which assigns to every $\D\in\Sh(\tau_\geq)$ the precosheaf $\D^\c\in\PcoSh(\tau_\leq)$ defined by the following commutative diagram
$$\xymatrix{\ \ (\tau_\geq)^{\op} \ar[r]^-{\D} & \V \\ \tau_\geq^{\c}\ar[u]^{(-)^{\c}} & \tau_\leq \,. \ar[l]_{\ \ \ \ =} \ar[u]_{\D^{\c}} }$$
\end{definition}

\begin{lemma} \label{Sheaves on posets are sheaves on Alexandrov}
Let $(P,\leq)$ be a poset and $\V$ a complete and cocomplete category where the sheaves and cosheaves under consideration take their values. For any $\D\in\Sh(\tau_\geq)$ the precosheaf $\D^\c$ is a cosheaf and the functor
$$(-)^\c:\Sh(\tau_\geq)\to\coSh(\tau_{\leq})$$
is an equivalence of categories making the following diagram commute
$$\xymatrix@C=2pc @R=2pc{ \Sh(P) \ar[d]_{\Lan} \ar[r]^-{=} & \coSh(P^{\,\op})  \ar[d]^{\Ran}  \\
\Sh(\tau_\geq) \ar[r]_-{(-)^{\c}} & \coSh(\tau_{\leq})}$$
\begin{proof}
It suffices to establish the commutativity of the diagram since $\Lan$ and $\Ran$ are equivalence of categories. To do so, consider the following diagram associated to any $\D\in\Sh(P)$
$$\xymatrix{
                                     &&      & P^{\,\op} \ar[dl]^{\D} \ar@/_2pc/[dlll]_{(-)^{\op}_\geq} \ar@/^2pc/[dddl]^{(-)_\leq} \\
(\tau_\geq)^{\op} \ar[rr]^-{\Lan \D} && \V   &         \\
&&& \\
\tau_\geq^\c \ar@{<->}[uu]^{(-)^\c}  && \ar@{<->}[ll]^-{\ \ \ =} \tau_\leq \ar[uu]_-{\Ran\D} \ar@{-->}@/^1pc/[uu]^-{(\Lan\D)^{\c}}. & }$$
Since $(\Lan\D)^\c\circ(-)_\geq=\Lan\D\circ(-)^\c\circ(-)_\leq=\Lan\D\circ(-)_\geq=\D$, the universal property of right Kan extensions ensures the existence of a natural transformation $(\Lan\D)^\c\to\Ran\D$. The inverse natural transformation is obtained similarly using the universal property of left Kan extensions and the functor $(-)^\c:\Sh(\tau_\geq)\to\coSh(\tau_\leq)$.
\end{proof}
\end{lemma}

\paragraph{Abelian sheaves over posets}
\begin{definition}(Projective objects)\label{Projective objects}
Let $\A$ be an abelian category. An object $\mathbb P\in\A$ is called \textbf{projective} if it satisfies any of the following equivalent conditions:
\begin{enumerate}
\item For any surjection $f:C\to B$ and any map $q:\mathbb P\to B$ there exists $g:\mathbb P\to C$ such that $q\circ g=f$, diagrammatically $$\xymatrix
{  & C \ar[d]^q \\
\mathbb P \ar@{-->}[ur]^g \ar[r]_f & B \ar[d]\\
   & 0.}\vspace*{-8pt}$$
\item Any exact sequence $$0\to A\to B\to \mathbb P\to 0$$ splits, i.e. it is isomorphic to $$0\to A\to A\oplus\mathbb P\to \mathbb P\to 0$$ with inclusion and projection maps.
%\item The functor $\Hom_{\A}(\mathbb P,-):\A\to\Ab$ is an exact functor, i.e. if $$0\to A\to B\to \mathbb P\to 0$$ is exact for any $B,C,D\in\A$, then $$0\to\Hom_{\A}(\mathbb P,B)\to\Hom_{\A}(\mathbb P,C)\to \Hom_{\A}(\mathbb P,D)\to 0$$ is exact.
\end{enumerate}
\end{definition}

\begin{remark}
The dual notion of an injective object will be omitted since it is not used in this work.
\end{remark}

\begin{definition}(Elementary projective sheaves and cosheaves) Let $\mathbb P\in\A$ be a projective object, $(P,\leq)$ a poset and $y$ an element in $P$.

The \textbf{elementary projective sheaf} $\mathbb P_{\leq y}\in\Sh(P,\Ab)$ with value $\mathbb P$ over $y$ is defined by
$$\mathbb P_{\leq y}[x]=\begin{cases}\mathbb P & \text{if } x\leq y \\ 0 & \text{if } x\nleq y, \end{cases}$$
with all non-zero morphisms equal to the identity.

The \textbf{elementary projective cosheaf} $\mathbb P_{y\leq }\in\coSh(P,\Ab)$ with value $\mathbb P$ over $y$ is defined by
$$\mathbb P_{y\leq}[z]=\begin{cases}\mathbb P & \text{if } y\leq z \\ 0 & \text{if } y\nleq z, \end{cases}$$
with all non-zero morphisms equal to the identity.
\end{definition}

\begin{terminology}A poset $(P,\leq)$ is said to be \textbf{locally finite} if for all pairs $x,z\in P$ the set $\{y\in P: x\leq y\leq z\}$ is finite.
\end{terminology}
\begin{lemma} \label{projective sheaves are local modules}
Let $\A$ be a cocomplete abelian category and $(P,\leq)$ a poset. A sheaf or a cosheaf over $P$ with values in $\A$ is projective if, and if $P$ is locally finite, only if, it is isomorphic to a direct sum of elementary projective ones.
\begin{proof}
Only the proof for sheaves will be presented since small variations adapt it for cosheaves. Notice that by the universal property of coproducts a direct sum of projective objects is projective. Explicitly, a morphism $\bigoplus P_i\to B$ defines a collection of morphism $P_i\to B$ by precomposing with the respective inclusion. given a surjection $A\to B$ one gets a collection of lifts $P_i\to A$ and therefore a lift $\bigoplus P_i\to A$.

A sheaf $\mathbb P_{\leq y}$ or cosheaf $\mathbb P_{y\leq }$ is projective since it is straightforward to check that
$$\Hom_{\Sh}(\mathbb P_{\leq y},\D)=\Hom(\mathbb P_{\leq y}[y],\D[y])$$
and
$$\Hom_{\coSh}(\mathbb P_{y \leq },\E)=\Hom(\mathbb P_{y \leq }[y],\E[y]).$$

Let $\P$ be a projective sheaf and $y$ an element of the poset.
Notice that by considering sheaves with support only on $y$ one concludes that $\P[y]$ is a projective object in $\A$.
By the local finiteness of the poset under consideration every set $\{z:z>y\}$ as a smallest element. Define the \textbf{near star} of $y$, denoted $\nst(y)$, as the sub-poset containing all such minimal elements. Notice that all pairs of elements in this sub-poset are not comparable.

Define the sheaf $\Q_y$ by
$$\Q_y[z]=\begin{cases}\P[z] & \text{if }z\in\nst(y) \\ \ \ 0 & \text{if }z\not\in\nst(y). \end{cases}$$
with all induced morphisms the zero map. The sheaf $\tilde{\Q_y}$ is defined by
$$\tilde\Q_y[z]=\begin{cases}\ \ \displaystyle{\P_y[z]} & \text{if }z\in\nst(y) \\ \ \ \ \ 0 & \text{if }z\not\in\nst(y)\text{ and } z\neq y\\ \displaystyle{\bigoplus_{z'\in\,\nst(y)}\!\!\!\P[z']} & \text{if }z=y\end{cases}$$
with non zero induced morphisms given by the inclusions $$\P[z]\to\displaystyle{\bigoplus_{z'\in\,\nst(y)}\P[z']}.$$
There are obvious surjections represented by solid arrows below
$$\xymatrix
{  & \tilde\Q_y \ar[d] \\
\mathbb \P \ar@{-->}[ur]^g \ar[r] & \Q_y \ar[d]\\
   & 0}$$
and the morphism $g[z]$ is the identity for all $z\in\nst(y)$ and it is zero for all $z\not\in\nst(y)$ such that $z\neq y$. The collection of maps $\P[z]\to\P[y]$ for $z\in\nst(y)$ gives a map $\bigoplus_{z\in\nst(y)}\P[z]\to\P[y]$ fitting into the following sequence
$$0\longrightarrow\!\!\bigoplus_{z\in\nst(y)}\!\!\P[z]\longrightarrow\P[y]\stackrel{g[y]}{\longrightarrow}\!\!\bigoplus_{z\in\nst(y)}\!\!\P[z]\longrightarrow0$$
which is exact since, denoting the inclusion $\P[z]\to\bigoplus_{z\in\nst(y)}\P[z]$ by $\iota_z$, for every $z>y$ one has
$$g[y]\circ\P[y<z]=\iota_z\circ g[z]=\iota_z.$$
The above sequence splits by the projectivity of the objects involved, see Definition \ref{Projective objects}, so there exists $B(y)$ such that $$\P[y]\cong B(y)\oplus\big(\!\!\!\bigoplus_{\,z\in\nst(y)}\!\!\P[z]\big)$$ iterating the argument one gets
$$\P[y]\cong \bigoplus_{z\geq y}B(z)$$ for every $y$ and therefore
$$\P=\bigoplus_{y\in P}B(y)_{\leq y}$$ concluding the proof.
\end{proof}
\end{lemma}

\begin{remark} \label{cocomplete for projectives}
The condition in Lemma \ref{projective sheaves are local modules} that $\A$ be cocomplete is in practice too restrictive. For example, the category $\Ab_{\f}$ of finitely generated abelian groups is not cocomplete. The conclusion of Lemma \ref{projective sheaves are local modules} remains true if one restricts to appropriate subcategories of sheaves or cosheaves on a poset where the relevant coproducts exist. Examples of such subcategories are the following.
\end{remark}

\begin{definition} (Sheaves and cosheaves with compact support) \label{sheaves and cosheaves with compact suport}
Let $P$ be a poset and $\A$ an abelian category. Denote by $\Sh_{\c}(X,\A)$ the full subcategory of $\Sh(X,\A)$ whose objects satisfy $\D[x]\neq0$ for at most finitely many $x\in P$. Define $\coSh_{\c}(X,\A)$ similarly.
\end{definition}

\begin{definition}(Enough projectives) An abelian category is said to have \textbf{enough projectives} if for any object $B\in\A$ there exists a surjection $$A\to B\to0\vspace*{-5pt}$$ with $A\in\A$ projective.
\end{definition}

\begin{lemma}Let $P$ be a poset and $\A$ an abelian category. If $\A$ has enough projectives then $\Sh_{\c}(P,\A)$ and $\coSh_{\c}(P,\A)$ do so as well. If in addition $\A$ is cocomplete, then $\Sh(P,\A)$ and $\coSh(P,\A)$ also have enough projectives.

\begin{proof}
Assume $\A$ is cocomplete and let $\D\in\Sh(X,\A)$. The first step is to construct a surjection of sheaves $\Q\to\D\to0$ with $\Q[x]\in\A$ projective for every $x\in P$. Choose for each $x\in P$ a surjection $f[x]:\Q[x]\to\D[x]\to0$ with $\Q[x]$ projective. For every pair $x\leq y$ there is a morphism $\D[y]\to\D[x]$ whose precomposition with $f[y]:\Q[y]\to\D[y]$ is represented by the horizontal solid arrow in the following diagram $$\xymatrix
{  & \Q[x] \ar[d]^{f[x]} \\
\Q[y] \ar[d]_{f[y]}\ar@{-->}[ur] \ar[r] & \D[x] \ar[d]\\
\D[y] \ar[ur]& \,\,0.}$$
The choice of a morphism realizing the dotted arrow for any pair $x\leq y$ makes $\Q$ into a sheaf and $f$ into a surjective morphism of sheaves. Define $\P\in\Sh(X)$ by $$\P[x]=\bigoplus_{x\leq y}\Q[y]_{\leq y}.$$ This sheaf maps surjectively onto $X$, therefore onto $\D$, and it is projective by Lemma \ref{projective sheaves are local modules}.

All other statements are proven in the same manner.
\end{proof}
\end{lemma}

\begin{notation} Let $\A$ be an abelian category. Denote by $\A_\bullet^{+}$ the \textbf{category of bounded below complexes} whose objects are homologically graded complexes which are zero below some degree. Notice that $\A_\bullet^{+}$ is enriched over $\Ab_\bullet$, and as usual one says that two morphisms are \textbf{chain homotopy equivalent} if they are homologous.
\end{notation}

The following are standard result in homological algebra, see for example \cite{Wei95} section 5.7 for their proofs.
\begin{lemma} \label{Projective resolutions and weak => strong for projectives}
Let $\A$ be an abelian category with enough projectives.
\begin{enumerate}
\item For any $A_\bullet\in\A_\bullet^+$ there exists a complex of projective objects $\P_\bullet\in\A^+_\bullet$ and a morphism $P_\bullet\to A_\bullet$ inducing an isomorphism in homology.
\item If $\P_\bullet$ and $\P'_\bullet$ are projective and $f:\P_\bullet\to\P'_\bullet$ induces an isomorphism in homology, then there exists $g\in\Hom_{\A_{\bullet}^+}(\P_\bullet,\P'_\bullet)\in\Ab_{\bullet}$ such that $f\circ g$ and $g\circ f$ are chain homotopy equivalent to the respective identities.
\end{enumerate}
\end{lemma}
\end{document}

